%!TEX root = monsterhunter.tex
\renewcommand*{\hbPartCover}{assets/ext/forest-sleeps}
%\renewcommand*{\hbPartSubcover}{assets/ext/lagiacrus2}
\part{For DMs}

\chapter{The Campaign}
\hbWideBottomArtFirstPage{1.98}{.98}{assets/ext/gore-fight}
I'm not telling you how to run your game. This book is provided with the information in it to make sure you can run just the game you want in the setting. Everything that I'm writing here are suggestions, some of which I don't even follow myself. I just hope that this book is able to reduce the work load of running your game in the \MH{} universe.

I would say that the ``default'' assumption is that the player characters are hunters, starting out at a low rank, and and climbing the hunter ladder through slaying monsters. Some advice about that:

\paragraph{The right tone.} \MH{} has a delightful duality of tone. The people in the cities are always happy, upbeat and jolly. They understand the threat of the monsters, but don't despair and keep a very happy outlook on life. Daily life outside the hunts should have a light, happy and whimsical tone, with side activities revolving about wacky antics of the characters or NPCs. Sometimes these wacky antics can result in a real quest for the players to embark one. Once that happens, the game kicks into a more serious gear. The monsters are believable and credibly dangerous. They don't pull punches and death is a threat which hunters face constantly (or maybe not, more on that later). The hunt is a real life-or-death situation and the players should feel that.

\paragraph{To Sandbox or Not to Sandbox.} This is a question I can definitely not answer for everyone, but I would say that the way the world of \MH{} is set up and the legacy of the video games sets it up for more of a sandbox game. The progression is always given by ranking up and facing stronger monsters. That doesn't mean that you can't tell your great story. But the whole point of collecting all the information in this book was to allow for the players to go off the rails, fight a different monster altogether because they felt like it. Just remember the players have as much to do with telling the story as you have.

\paragraph{More than quests.} The above might be a fine basic premise for a campaign but it is not nearly enough to carry a game. Enrich the game with the NPCs that are ancillary to the hunters' career, such as the craftsmen, traders, cooks, and caravaneers that the players work for and with. If you are doing a good job, the encounters with the monsters will be memorable, but the world of monster hunter has so much more to offer than monsters. It is whimsical, silly and full of fun characters. Fill the cities, villages and caravans, even the wilderness with such characters. Alternate between quests and downtime so the players have a chance to use something other than their weapons and explore their characters.

\hbWideBottomArtFirstPageFix
\paragraph{More than just a quest.} Enrich the quests the players get by giving them meaning and background, as well as more stages than to go to the monster and killing it. For example: A quest might be given without explicitly stating what monster is to be hunted. The players may need to travel to the place it was last seen and find clues, such as tracks, fallen scales or the prey it slew, as well as question witnesses to get an idea of what they're up against. Once that is done, they can make preparations to fight the monster or monsters in the most favorable location and the best equipment (more on that below). The buildup can go a long way to making the monster relevant (one way of tracking if the players are taking the monster seriously is to see if they remember its name).

\section{On the Hunt}
There are probably many different ways we can structure a hunt. Some DMs and groups might prefer to model it as a gruelling survival experience, overcoming weather and the forces of nature before even facing the monster. Personally, I like to keep it relatively focused on the actual monster hunting part and not focus too much on the survival aspect. But pointing out the various ways in which an environment challenges the players' survival can help making one area feel distinct from others that the players have been in. Giving character to the wilderness is essential, as the players will likely be spending a lot of time there.

\paragraph{The Base Camp.} The Hunters' Guild has organised the profession of hunting considerably and made arrangements to make a successful hunt more likely. As such, there will usually be a guild-maintained base camp close to the area of a quest. A base camp contains no supplies, unless the hunters specifically arrange to have supplies dropped off there before they arrive. It does contain beds, cooking and crafting equipment to maintain or make basic repairs to weapons, armor and tools as well as a large chest that the hunters can store their belongings in if they are not needed during the actual hunting. The base camps are usually cozy, well-protected places and often have a very beautiful view.

%death or carting

\subsection{Giddyup}
\textit{(From reddit user u/famoushippopotamus) A bit harsher than it needs to be but it gets the point across.}

Where are we, when our players are crawling through the dark, claustrophobic nightmares that we have put them?\par
Are we side-by-side with them? Breathing the same stale air? Hearing the same chitinous scrabbles in the echoing tunnels?\par
Or are we above it all. Gazing down as a beneficent overlord, our x-ray vision seeing through rock and stone, to judge and test our mortal prey?\par
I'll tell you where you should be.\par
Down there in the dark.\par
If you aren't scraping your knees and choking on wood fumes from your sputtering torch; if you aren't terrified your wounds are going to be hosting a new and interesting flesh-eating fungus; if you aren't wondering if you are going to have to ditch your long bow in order to squeeze down that crack, then you aren't doing your game a good turn.\par
You need to feel it. You need to live it. Right with the party. You need to feel the same terrors. The same doubts. The same flights of panicked fancy.\par
How else do you expect to convey the vision of what you do if you aren't living it? What's that old axiom? "Write what you know"?\par
In D\&D sometimes that's impossible. None of us have ever cast a fireball. Or wrestled an Ogrillion. Or traveled the astral plane.\par
But all of us, every day, have struggled with fear. With anxiety. With the dawning dread that we might not be up to the tasks ahead of us. All of us have fought for our emotional survival. Some of us have battle scars. Big ugly twisted ones.
The unknown is what drives the adventurer. To push themselves and to wonder why on Gygax's green earth he or she is miles below the sun, in the muck, in the shit, in the cold running water, with hungry things all around them.\par
We struggle in the dark because we want to bring light into shadow. To show the hungry dark that we are not afraid. That we are going to overcome, no matter the odds.
Recover the relic. Slay the demon. Shut the gate. Rescue the princess. Defeat the army. Kill the assassin. Trick the dragon. Rally the troops. Fight the fight.\par
Victory. Or death.\par
This is why we DM. To bring both in equal measures.\par
But How?\par
\subsubsection{Well. I'll Tell You.}
\hbBottomRightArt{.706}{.9}{.85}{assets/ext/najarala-art}
Its time to get serious about what we are doing. You want to "become a great dungeon master"? Like your heroes whose names you intone like mystical words of power? Perkins. Mercer. Gygax.

Then you need to do what all those great DMs did. What the current crop of great DMs are doing. Right now. Day in. Day out. You know who you are. We recognize one another.

There is dirt on our elbows. There are scars across our knuckles. There are aches and pains that cannot be seen. We've all walked the same battlefields. Seen the same wars.

You want to be a great DM? Then you need to get dirty. You need to be at the table. Week in and week out. Making mistakes.

Let me repeat that.

Making mistakes. Big ones. Campaign ending ones. Ones that your friends make fun of for years.

You need to get dirty. You need to get your hands bloody with the deaths of your friends. Get right with death. Get right with failure. Get right with the \emph{idea that you are going to suck for a long time because nothing comes easily.}

So many new DMs that I see wanting to run a "great game where everyone has fun, I have a grasp on the rules, and everyone wants to come back and they think its amazing."

Pardon me, but are you drunk? \emph{No one} runs games like that when they start. They are \emph{howlingly} bad. If I were to show you the campaign notes from my first few sessions you wouldn't be able to look me in the eye anymore.

You need to get dirty. You need to work. Hard. Every week. You need to create and discard all kinds of insane-rules-that-you-made-up-because-yeah-that-sounds-awesome-but-ultimately-is-broken-crap. You need to \emph{get used to fucking up}. Embrace it. Welcome it. Take it for what it is\hbNone a lesson to be learned.

You need to screw up encounter after encounter, because CR sounds nice on paper but its not worth the ink that printed it when it comes to the heat of the battle. \emph{There is only one way to learn to create meaningful encounters, and that is to screw up hundreds of them first.}

Experience is the best teacher. Not online tools. Not blog posts. Not reddit. \textbf{Get out there and do it}.

Just like real life, you'll get better as you go, if you are paying attention. If you care to get better, you will.

I've been a DM for nearly 3 decades and \emph{every, single game session} I make a mistake. Every time. It might be something small, or it might be something big, but I have learned to not only expect the mistake, but to welcome it. That mistake is a lesson learned. And I try not to make that mistake again, although sometimes I do, and that's ok. Sometimes we need to get cracked across the face a few times before it sinks in.

Make mistakes. Work hard. Get dirty.

Saddle up.

\chapter{The Monsters and Game Balance}
I don't provide stat blocks for the monsters. The main reason for this is that I have not zeroed in on the balancing of the monsters yet, so I don't want to commit to stat blocks. Of course, we do need stats for them during a fight. Of course, you can develop your own stat blocks. I suggest using the monster stat table below as a base for developing the monsters. It was originally developed by reddit user \texttt{u/gradenko\_2000} and can be found at \url{https://www.reddit.com/r/dndnext/comments/2wk4in/improving_the_monster_quick_stats_table/}.

The basic assumption that went into the table is this: A monster should die from four hits from each of the four party members. A ``hit'' in this case means an attack action where all attacks hit. A player has a 60\% chance of hitting an appropriate-level monster, the monster has a 40\% chance of hitting that player. The original post then goes on to estimate the amount of damage a fighter will do per round at a given level. Similarly, a monster is supposed to be able to kill a player character with four successful attack rounds.

Having run several fights using this table, I find it produces monsters that are far too easy. Your party might not be made up of hard-core gamers, but even first-time 5e players just using their class abilities were able to make short work of the monsters. That's not what we want. There are issues with the original table, such as assuming a fighter using a one-handed weapon but no shield, or disregarding magical weapons which (drastically) increase damage per round, as well as magical buffs, which often render these supposed beasts completely powerless.

In addition, monsters in \MH{} aren't supposed to go down in one fight. In the video game, they rarely have trouble getting away. In \DND{}, it's not so simple. Ranged attacks are far more common in D\&D than they are in monster hunter and a fleeing monster will take numerous hits while it's trying to get away.

\subsubsection{Hit Points}
So how do you set up the monster's hit point total, taking into account that it is supposed to survive multiple encounters? Here are some (not mutually exclusive) thoughts:
\paragraph{Determine the Monster's total hit points ahead of time.} Record the actual damage the party does per round, if all attacks hit. Multiply by four, for the hit point total of one encounter. After that, the monster should consider fleeing. Then multiply again by the number of encounters with the monster. This is the monster's total.
\paragraph{Work out the hit points per-encounter.} As above, but the monster flees after its hit points for one encounter are used up, and excess damage is not counted.
\paragraph{Monster's hit points don't all come from hit dice.} After a fight with the monster, the party as well as the monster may decide on taking a short rest to refill hit points. If the party started with full hit dice, they are expected to be able to heal up to full hit points. If the monster can do that as well, there is not really any point to the rest, as the fight would just start from square one. So either the monster regains no hit points, or only some.
\paragraph{Increase the monster's attack, not hit points.} Maybe it should be the party that has to retreat, not the monster. For this, you should know your group, as some groups react very negatively to the idea of retreating from a fight (it feels like losing to them) and going back in later. If done right, this can be a more exciting fight, as the monster is less of a hit point sponge and really actively trying to kill the party without being killed, rather than resting on the assurance that it has plenty of hit points left.

I have not mastered the art of balancing monster fights, and as described in the text above by \texttt{u/famoushippopotamus}, expect to fail many times before getting it right.

\subsubsection{Damage Per Attack}
In addition to the monster's hit points, we also need to figure out how much damage it should do and what it's attack bonus is. From experience, I can say that you could to expect the front-line fighter to have something like 23 AC at level 2, or as soon as they have access to plate armor (18 from plate armor, +2 from shield, +2 from \emph{shield of faith}, +1 from \emph{defense} fighting style). Not every party will know or make use of all of these features, of course, but it is possible. Personally, I rule that AC bonuses from spells do not stack, so as not to let things get out of hand. This figure will not really increase much, either, until the party begins to make use of armorskins or magic armor and shields, several levels later. There is a difficult line between rewarding the highly armored fighter with a low chance of being hit and the monster ``wasting'' its attacks on the front line where it is unlikely to hit.

Perhaps this is a moment to take inspiration from the video games: The monsters cycle through targets relatively evenly, and due to their high mobility, will usually be able to hit whoever they want to. Unless there is a specific feature being used to draw the attacks of the monster to the toughest enemy (such as \emph{goading attack}, \emph{compelled duel} or \emph{enthrall}), they will spread out their attacks relatively easily. That way, the fighter can be rewarded for having higher AC while also making it a less attractive strategy to pile all buffs on the strongest character.

\subsubsection{Attacks Per Round}
\hbWideTopArt{1.98}{.95}{assets/ext/chameleos-art}
Another point about buffs in 5e is that practically all of them require concentration. The monsters may or may not understand magic, but the mage should have to take an attack every now and again, preferably one with a saving throw for some guaranteed damage and a concentration check. \emph{Haste} is incredibly powerful, but devastating if it fails.

Four hits to kill still seems like a good place to start for monster damage, but I would definitely interpret that as four actual hits, not four full attack actions (which may be as much as 12 or 16 hits). Which raises the question of how many hits a monster should do per round. As how many hits do we count an area-of-effect attack? Very likely, it will hit multiple targets with it the first time, but far fewer after that, as the party spreads out.

Before figuring that out, let's consider another factor: Action Economy. There is (usually) one monster to around four PCs. Disregarding the ways the PCs have of making the monster waste its action (more on that below), it means the monster has one action for every four that the players have. It is expected to be felled after 27 attack actions by the players (sixteen need to hit, but only 60\% of them do), or about seven turns (this, in my opinion is actually a good length for a combat). That means it gets between six and eight actions. Cycling between targets every round, each player will get one to two facefulls of monster.

That's the data: Six to eight actions for your monster, four hits kill a player, hit chance is about 40\% for the fighter (not counting buffs), more for the squishy back line. Three attacks seem like a good place to start then. Maybe two, maybe four sometimes. Count area-of-effect as one and a half at most. Count attacks with saves as one and a half as well if they deal damage on a successful save as well. That's why I'm not making stat blocks. I don't want to write in stone what and how many attacks the monster has. Sometimes, it's going to go all-out and have more, sometimes the way it's positioned just means it has fewer.

\subsubsection{Spicing It Up}
But don't let the fight turn into just an exchange of ordinary attacks. Find other ways of having the monster get in damage, such as automatic defenses (like \tableicon{deviljhoS}~Deviljho's saliva) or get creative with reactions. Experiment with legendary actions, moving some small, ordinary attacks into the legendary actions and use the actions for the big signature moves.

Speaking of signature moves: If you have played \MH{}, you are probably already familiar with many of the monsters' iconic moves. If not, in the monster description I have at least tried to allude to most of the ways the monsters have of inflicting pain on you.

Here are some more ways of spicing up your monster's attacks:
\paragraph{Status Effects.} I would usually tie the effects to one of the conditions from the \PHB{} and word them like \emph{you are poisoned for 5 rounds. As long as you are poisoned\ldots} In addition to providing unique flavour to a monster's attacks, they can provide crowd control, preventing a player from attacking for a round to make up for the players' damage output, which is invariably higher than you expected. Status effects can range from poison to being covered in mud, snow or poop, being on fire, getting dusted with explosive dust, getting frozen, drenched or shrunk. The different status effects can also have very different ways of getting rid of them. Getting rid of a non-magical status effect should require at least an action. Magical status effects usually allow for saving throws at the end of the player's turn.
\paragraph{Displacement.} Attacks from these big monsters might fling the players around. Getting knocked around usually requires a Strength saving throw, though if the original attack already had a saving throw, it should probably not use another. Consider scaling the distance covered by how much the saving throw failed.
\paragraph{Grapple.} Some attacks might pin a target, leading to it being grappled. In these cases, the attacker usually gets to make another attack as a bonus action. Grapple can really eat a character's hit points quickly, especially if it coupled with an effect like restrained. A grapple is usually released if the monster is hit by a dung bomb and fails a Wisdom saving throw.
\paragraph{Semi-Legendary actions.} Some actions might be quicker than a regular action, but slower than a legendary action, or might happen several times throughout a turn (an example would be the three to five charge attacks of \tableicon{tigrexS}~Tigrex). In this case, reduce the initiative of the monster by some amount when it uses the attack, then resume on the new count, which will be in the same round. If the initiative would be reduced below zero, instead increase it to 20 plus the monster's initiative bonus plus the monster's current initiative minus the amount it is reduced after the attack.

\subsubsection{Disabling}
One of the most problematic things to balance are effects which completely disable a monster, such as \emph{bestow curse} or paralysing poison. As we have done before, here are my thoughts on it:
\paragraph{Poisons need more than one dose to apply effects.} Applying poison to your weapon is an action. A successful attack applies one dose of poison to the monster. After that, it will need to be applied again. I find three to be a good number of doses that the poison needs to apply. For ranged weapons, it takes an action to switch to poison ammo. After that, each attack carries the poison and every two successful attacks apply one dose of poison. You do not need additional actions between shots, you have switched to your poison ammo belt (heavy bowgun) or magazine (light bowgun).
\paragraph{Avoid the paralysed condition.} Making every attack against the monster a crit is very likely to trivialise the fight. That should never happen. Even if the poison paralysed when the players got it (such as from a \tableicon{bnahabraS}~Bnahabra or a \tableicon{genpreyS}~Genprey), it should only inflict the stunned condition to a monster.
\paragraph{Whatever happens, it only lasts for one round.} Under no circumstances should a disabling effect last for more than one round. A spell is automatically resisted after the first round, the monster escapes from the trap after one round, the poison lasts for one round, etc. This doesn't mean that the players can't chain multiple disabling effects, just that each one will only cost the monster one action (that is significant enough).
\paragraph{Resist during legendary actions.} Alternatively, if you're using legendary actions, consider how they are affected by the effect and let the monster use them to make saving throws to shake off the effect.

\subsubsection{The Table}
Well, here's what you've all been waiting for, the monster stat table. Here's how you read it:

\paragraph{Stars.} The monster's rating, like given in the monsters chapter. One star could be said to correspond to two CR.
\paragraph{AC.} The monster's armor class. All monsters have natural armor. In some cases, the monster might increase its armor class by adding some sort of hard shell, like \tableicon{zamtriosS}~Zamtrios encasing itself in ice.
\paragraph{HP.} Depending on how you deal with the monster's hp, this is a suggestion for the hp for one encounter.
\paragraph{To Hit.} The monster's to-hit bonus. It is usually the same for all of the monster's attacks, possibly different for melee and ranged attacks. A monster will usually have a reach of at least 10 feet.
\paragraph{DPA.} Damage Per Attack. As described above, this is the average damage that a monster should deal with each attack. Every monster has stronger and weaker attacks. Do not roll more than two dice for a physical attack. Roll only dice (no constant bonus) for special attacks such as fireballs.
\paragraph{Saves.} Given as best save/good save/bad save. Most monsters should have one best save, two good saves and the rest bad saves.
\paragraph{DC.} The save DC for a monster's abilities. It's generally the same for all abilities. Consider carefully how many abilities should have a saving throw attached, as too many of them will make the players feel that their AC is irrelevant.

Adjust up and down, as well as season, to taste.

\begin{hbNarrowTable}{Monster Statistics}{CCRCZCC}
\textbf{Stars} & \textbf{AC} & \textbf{HP} & \textbf{To Hit} & \textbf{DPA} & \textbf{Saves} & \textbf{DC}\\
1              & 13          &   120       &  +3             &   5          &  5/3/1         & 13\\
2              & 14          &   136       &  +5             &   9          &  6/3/1         & 14\\
3              & 15          &   272       &  +7             &  13          &  7/4/2         & 15\\
4              & 16          &   336       &  +8             &  17          &  8/5/2         & 16\\
5              & 17          &   400       &  +8             &  21          &  9/5/2         & 16\\
6              & 18          &   600       &  +9             &  28          &  9/5/2         & 17\\
7              & 19          &   792       &  +9             &  32          & 10/6/3         & 18\\
8              & 20          &   850       & +10             &  41          & 10/6/3         & 18\\
9              & 21          & 1,056       & +10             &  46          & 11/7/3         & 19\\
10             & 22          & 1,320       & +11             &  56          & 11/7/3         & 19\\
\end{hbNarrowTable}

There. There is no way I could possibly be horribly wrong, is there?

\chapter{Random Tables}
\section{Building a Random Quest Board}
So your adventurers are looking at a quest board and you don't know what's on it. Well, first, determine the number of quests available. This is dependend on how remote the location of this board is. A more remote location will usually have fewer quests available. Also, the quests won't change as frequently. Remove about a third of the quests available and add about as many new ones.

\begin{hbNarrowTable}{Quest Quantity}{YCC}
\textbf{Type of Location} & \textbf{Amount} & \textbf{New Quests\ldots}\\
Major Guild Hub & 3d10 & every 12 hours\\
Major Guild Outpost & 2d10 & every day\\
Minor Guild Outpost & 2d8 & every two days\\
Large Gathering Hall & 2d6 & every week\\
Small Gathering Hall & 2d4 & every two weeks\\
\end{hbNarrowTable}

Next, determine the level of the quest. Roll 2d10 and drop the higher result. The result is the number of stars of the quest.

Next, determine the type of quest. Roll 1d10. If it is higher than twice the level of the quest, the quest is a gathering quest, otherwise it is a hunting quest.

Next, determine the target of the quest. There is a 30\% chance that the target is unknown (determine the target in secret). Gathering quests also have a target, though it is optional and it may also be unknown. To determine the target, randomly determine a monster from the monsters by level table in the monsters chapter that is of the same level as the quest.

\begin{hbNarrowTable}{Random Weather Table}{RYYL}
\textbf{d100} & \textbf{Cold Climate} & \textbf{Temperate Climate} & \textbf{Desert}\\
01-70 & Cold, calm & Normal for season & Hot, calm\\
71-80 & Heat Wave (71-73) or cold snap (74-80) & Heat wave (71-75) or cold snap (76-80) & Hot, windy\\
81-90 & Precipitation (snow) & Normal precipitation for season & Hot, windy\\
91-99 & Snowstorm & Thunderstorm, snowstorm & Duststorm\\
100 &   Blizzard & Windstorm, blizzard, hurricane, tornado & Downpour
\end{hbNarrowTable}

\section{Random Encounter Tables}
These used to be in the wilderness chapter, but I have decided to hide them here, away from the peeking eyes of players.
\begin{description}
  \item[This is a stest] slfjsdklfjskldjflskdjf
\end{description}

\begin{hbNarrowTable}{Forest/Jungle Random Encounters}{RY}
\textbf{d100} & \textbf{Encounter}\\
 & \tableicon{bnahabraS} Bnahabra$\times$2d4\\
 & \tableicon{jaggiS} Jaggi/\tableicon{genpreyS} Genprey$\times$2d4\\
 & \tableicon{ludrothS} Ludroth/\tableicon{maccaoS} Maccao$\times$1d4+1\\
 & \tableicon{melynxS} Melynx/\smallicon{assets/ext/iconsL/shakalaka} Shakalaka\\
 & An exploration party attempting to map the forest.\\
 & \tableicon{great-jaggiS} Great Jaggi/\tableicon{gendromeS} Gendrome\\
 & \tableicon{bulldromeS} Bulldrome/\tableicon{kecha-wachaS} Kecha Wacha\\
 & \tableicon{royal-ludrothS} Royal Ludroth/\tableicon{great-maccaoS} Great Maccao\\
 & \tableicon{arzurosS} Arzuros/\tableicon{congalalaS} Congalala\\
 & Hunters searching for an unknown monster\\
 & \tableicon{qurupecoS} Qurupeco/\tableicon{gypcerosS} Gypceros\\
 & \tableicon{rathianS} Rathian/\tableicon{malfestioS} Malfestio\\
 & \tableicon{rathalosS} Rathalos\\
 & \tableicon{duramborosS} Duramboros/\tableicon{glavenusS} Glavenus\\
 & \tableicon{nargacugaS} Nargacuga/\tableicon{najaralaS} Najarala\\
 & \tableicon{zinogreS} Zinogre\\
 & \tableicon{deviljhoS} Deviljho/\tableicon{rajangS} Rajang\\
 & \smallicon{assets/ext/iconsL/yama-tsukami} Yama Tsukami/\tableicon{chameleosS} Chameleos
\end{hbNarrowTable}

\begin{hbNarrowTable}{Grassland Random Encounters}{RY}
\textbf{d100} & \textbf{Encounter}\\
1-5 & \tableicon{vespoidS} Vespoid$\times$2d4\\
6-10 & A group of \tableicon{mosswineS}~Mosswine excitedly digging for something. When investigating, you find several \tableicon{mushroom-white}~Choice Mushrooms and a charm with the Mycology ability.\\
11-18 & You discover tracks. By following them or examining with your knowledge of monsters, you learn that they from a \tableicon{bulldromeS}~Bulldrome, which may attack.\\
19-25 & You discover a well-hidden ruin built by the ancient civilisation.\\
26-33 & \tableicon{velocipreyS}~Velociprey$\times$2d4.\\
34-43 & You come across several \tableicon{melynxS}~Melynx on the prowl.\\
44-53 & \tableicon{velocidromeS}~Velocidrome\\
54-60 & You are overtaken by a member of the \tableicon{felyneS}~Felyne mail, mounted on a \tableicon{gargwaS}~Gargwa\\
61-68 & \tableicon{yian-kut-kuS}~Yian Kut-Ku\\
69-75 & You come across another hunter party, who are on a quest of their own.\\
76-80 & 2d10 stampeding \tableicon{aptonothS}~Aptonoth\\
81-85 & \tableicon{rathianS}~Rathian\\
86-90 & Near a stream, a caravan has become stuck in the mud.\\
91-93 & You smell smoke. It is coming from a bushfire some distance away. There is a 50\% chance it was caused by a monster.\\
94-95 & \tableicon{deviljhoS}~Deviljho\\
96-98 & From some distance away, you spot the lumbering figure of \smallicon{assets/ext/iconsL/lao-shan-lung}~Lao-Shan Lung.\\
99-100 & You catch a glimpse of \tableicon{kushala-daoraS}~Kushala Daora, who is preceeded by a massive thunderstorm.\\
\end{hbNarrowTable}

% Desert:
% Bnahabra, Cephalos, Remobra, Ioprey, Desert Seltas+Queen, Daimyo Hermitaur, Diablos/Monoblos, Glavenus, Seregios, Rathian, Teostra, Volvidon, Kushala Daora
