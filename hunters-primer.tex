%!TEX root = monsterhunter.tex
\renewcommand*{\hbPartCover}{assets/ext/stand}
\renewcommand*{\hbPartSubcover}{assets/ext/stand2}
%XXX Off-center part, probably due to twocolumn acting up. Or openany
\part{Hunter's Primer}

\chapter{Hunter's Arsenal}
To overcome the many monsters that inhabit the wilderness, hunters rely on a variety of different weapons and tools that give them an edge over the bigger, faster and stronger monsters.

In addition to the equipment here, an often underappreciated item in the hunter's arsenal is information. Through observation, a group of hunter can study the weaknesses of a monster before facing it.

\section{Weapons}
The Guild has a range of suggestions for hunter on a variety of weapons to use. While they recommend the use of one of these weapons, they are not all that is available. Some of these differ from the \PHB{} or are completely new. All weapons are martial weapons unless designated simple.

\imageheader{assets/ext/iconsL/sword-and-shield-white}{Sword \& Shield}{Damage, cost and weight vary}
A common, fast, bread-and-butter combination ideal for many hunters who choose to travel light, but do not want to sacrifice defensive potential. Typical sword choices are the longsword or rapier from the \PHB.

\imageheader{assets/ext/iconsL/dual-blades-white}{Dual Blades}{Damage, cost and weight vary}
Hunters wishing for more fast damage output at the cost of defence sometimes choose to wield two blades. One advantage of dual wielding is being able to utilise two different special weapon properties. Typical choices of swords are the scimitars and/or shortswords from the \PHB.

\imageheader{assets/ext/iconsL/great-sword-white}{Great Sword}{1d12 slashing, 50gp, 30lb., Heavy, two-handed}
A truly colossal weapon, the Great Sword swings slow but hits hard. A well-placed hit from this weapon can deal devastating damage. It is the same as the greataxe from the Player's Handbook.

\imageheader{assets/ext/iconsL/long-sword-white}{Long Sword}{1d10 slashing, 20gp, 18lb., Heavy, reach, two-handed}
Not to be confused with the longsword from the \PHB, the Long Sword is a typically curved sword about 6 feet long. The key to mastering it is positioning yourself to strike with the strongest part of the blade for best damage. Any character who is proficient with greatswords is also proficient with the Long Sword.

\imageheader{assets/ext/iconsL/hammer-white}{Hammer}{2d6 bludgeoning, 10gp., 30lb. Heavy, Two-handed}
A weapon with similar weight as the Great Sword, it brings this weight to bear directly as blunt damage. It is the same as the maul from the Player's Handbook.

\imageheader{assets/ext/iconsL/hunting-horn-white}{Hunting Horn}{1d8 bludgeoning, 5gp, 20lb., Heavy, two-handed, Simple}
The Hunting Horn encompasses a class of blunt melee weapons with a musical instrument built in, usually a horn or bagpipes. To use it as a weapon, one has to be proficient with greatclubs. To use it as a musical instrument, one must be proficient with the instrument that is built in. A popular choice for bard hunters.

\imageheader{assets/ext/iconsL/insect-glaive-white}{Glaive}{1d6 slashing, 40gp, 8lb., Double}
A Ranger specialty, the Glaive is a double-ended weapon. Wielding it is similar to dual wielding two light weapons and requires two hands (though the two attacks are one-handed). Rangers are proficient with the Glaive. It is traditional to keep an insect animal companion (a kinsect) in addition to wielding this weapon, leading to its more common name: Insect Glaive.

\imageheader{assets/ext/iconsL/lance-white}{Lance}{1d8 piercing, 10gp, 12lb., Reach, special, versatile (1d10)}
The Lance is a weapon intended for use on foot and optimized for charging. Attacks you make with a Lance against targets 5 feet away have disadvantage. If you have the Charger feat, you may choose to both shove and attack when you take advantage of the feat.

\imageheader{assets/ext/iconsL/gunlance-white}{Gunlance}{1d6 slashing, 15gp, 18lb., Heavy, ammunition (range 30/90), special}
The Gunlance is a fusion of melee and ranged weapon but it can also use its shelling capacity to augment melee attacks. When used for a ranged attack, it deals 1d4 piercing damage. When hitting with a melee attack, the wielder may use a bonus action to make a ranged attack with advantage against the target of the melee attack (ignoring the usual penalty for using a ranged weapon within 5 feet of the target).

\imageheader{assets/ext/iconsL/switch-axe-white}{Switch Axe}{1d8 slashing, 25gp, 24lb., Heavy, two-handed, versatile (1d10), special}
The Switch Axe can toggle between two modes: sword mode and axe mode. In sword mode, the axe head is used as a shield (for the usual shield benefit). In axe mode, the weapon is a two-handed greataxe. Switching between the modes is a bonus action. Its main drawback is that the shield is integrated with the weapon and cannot be replaced with a magically enhanced one.

\imageheader{assets/ext/iconsL/light-bowgun-white}{Light Bowgun}{1d6 piercing, 25gp, 8lb., Ammunition (range 80/320), simple}
Similar to a crossbow, but firing crude gunpowder shots and with a special loading mechanism, the Bowgun represents the state of the art of hunter weaponry. Stats-wise, it is equivalent to the shortbow from the \PHB.

\imageheader{assets/ext/iconsL/heavy-bowgun-white}{Heavy Bowgun}{1d10 piercing, 50gp, 25lb., Ammunition (range 100/400), heavy, two-handed, loading, special}
The Heavy Bowgun is very unwieldy when used while mobile. As a bonus action, you can set down or pick up the Heavy Bowgun. You may only do this if you do not move on that turn. While manning the deployed Bowgun, you are immobile (gaining the grappled condition) and the weapon loses the loading property.

\imageheader{assets/ext/iconsL/bow-white}{Bow}{1d8 piercing, 40gp, 6lb., Ammunition (range 150/600), heavy, two-handed}
The Bow is the ranged weapon of choice when wishing for a heavy weapon without sacrificing mobility. It is the same as the longbow from the \PHB.

\imageheader{assets/ext/iconsL/boomerang-white}{Boomerang}{1d4 bludgeoning, 2gp, 1lb., Thrown (range 20/60), simple}
A weapon favoured by Lynians, the Boomerang is a thrown weapon that can be surprisingly deadly if thrown by a skilled fighter. If you have proficiency, the Boomerang returns to you and can be thrown again next round. If you are able to make multiple attacks, you must carry that many Boomerangs.

\imageheader{assets/ext/iconsL/blunt-white}{Felyne Catspaw}{1d8 bludgeoning, 1gp, 8lb., Two-Handed, simple}
A traditional Felyne weapon, often wielded by Melynx. It is similar to a mace where the head is a large stylised catspaw.

\section{Heavy Weapons}

Many installations built to defend against monster attacks, as well as many sand-, water- or airships have built-in stationary heavy weapons. While cumbersome to operate, these pack more of a punch than any weapon that can be carried. These weapons can turn the tide even in a fight against the mighty Elder Dragons.

\imageheader{assets/ext/iconsL/ammo-white}{Ballista}{range 120/480 ft., 3d10 piercing}
A common heavy weapon due to its light weight and great versatility. It usually has a wide firing arc, but comparatively low damage. It takes an action to load and an action to fire, though they need not be carried out by the same person. Due to its light build, a ballista is fairly easily destroyed, should it be hit by a stray fireball.

\imageheader{assets/ext/iconsL/ammo-yellow}{Ballista Binder}{range 80/320 ft., special}
A special type of ammo that can be shot from almost any ballista. The point of this type of shot is not to wound, but to entangle the monster and make it vulnerable to other attacks. Treat this as the restrained condition. If the monster was flying when it was hit, it now falls to the ground. On subsequent turns/actions, it can struggle against the bindings.

\imageheader{assets/ext/iconsL/ammo-red}{Cannon}{range 300/1,200 ft., 8d10 bludgeoning}
The usual choice for heavy bombardment. These weapons are best operated by multiple people as it takes at an action to load it (in addition to fetching the ammunition) and one to fire it. They are usually mounted in a fixed direction and cannot be aimed. However, their high damage makes up for their cumbersome nature.

\imageheader{assets/ext/iconsL/coin-white}{The Dragonator}{reach 20 ft., 10d10 piercing, double damage against dragons}
The Dragonator is the most destructive weapon ever designed in the fight against monsters. Developed at a time of great crisis, it has singlehandedly turned fights against the Elder Dragons in the humans' favour. It is always mounted in a fixed position and its spring-loaded construction is triggered at a seperate switch. It is useful against any kind of monster, but the magic with which its metal construction is infused makes it particularly effective against dragons.

The fact that the Hunters' Guild condones the use of this weapon is not without controversy. It is a vile weapon, designed to inflict cruel and debilitating wounds in order to cause a dragon (its intended target) to flee. In addition, the magical enchantments placed on each Dragonator are fueled from a deep hatred for the dragons and the terror they represent. This is in contrast to the Guild's founding principle of \emph{respect for nature}. However, the use of the Dragonator has proven necessary again and again when cities and caravans were threatened by the seemingly invincible Elder Dragons. At least, its use against non-dragons is discouraged.


\section{Tools}
In addition to causing damage, a variety of other effects can be achieved with a clever application of available materials. This list is by no means exhaustive and hunters are always encouraged to attempt new tricks to give them an edge against a monster.

\imageheader{assets/ext/iconsL/sac-purple}{Poison}{}
There are many different poisonous plants and mushrooms, as well as venomous monsters. Poison can be applied in various ways, whether by making tinged meat~\smallicon{assets/ext/icons/meat-purple}, applying it to your weapons, making poisoned daggers~\smallicon{assets/ext/icons/knife-purple} or spreading it as a cloud with a poison smoke bomb~\smallicon{assets/ext/icons/smoke-purple}. Keep in mind however, that as many of the monsters that hunters often face are quite large, they will usually need more than one application of poison for it to be effective.

\imageheader{assets/ext/iconsL/trap-blue}{Traps}{Usually pitfall~\smallicon{assets/ext/icons/trap-green} or shock~\smallicon{assets/ext/icons/trap-purple} traps}
The purpose of a trap is to temporarily immobilise a monster to make it vulnerable to attacks or prevent it from attacking. Usually, a trap will not be able to contain a large monster for a long time, so it would be wise to make good use of the time bought.

\imageheader{assets/ext/iconsL/monster-jewel-yellow}{Flash \& Sonic Bombs}{} Faster to apply but also shorter in effect than traps are flash~\smallicon{assets/ext/icons/monster-jewel-yellow} and sonic~\smallicon{assets/ext/icons/monster-jewel-grey} bombs. They can be thrown like grenades and will explode for the desired effect. Proper research on the monster can determine which type of bomb is effective and when it is best used, as timing is of the essence with bombs. Just make sure your fellow hunters are safe!

\imageheader{assets/ext/iconsL/dung}{Dung Bombs}{}
On the defensive side, a popular option for getting out of a tight spot is the timely application of a dung bomb. Many monsters will let even a tasty hunter slip through their fingers when confronted with a steaming pile of poo to the face. Again, research is important as not every monster will react the same and it would be a shame if this plan would blow up in your face\ldots

\imageheader{assets/ext/iconsL/smoke-white}{Smoke Bombs}{}
Whether to make a quick escape or a covert entrance, smoke bombs can give concealment but can also be suspicious if used incorrectly. Using an action, it fills a sphere with a 20 foot radius with a thick fog cloud. The sphere spreads around corners, and its area is heavily obscured. It lasts for 5 minutes or until a wind of moderate or greater speed (at least 10 miles per hour) disperses it.

\imageheader{assets/ext/iconsL/barrel-bomb-red}{Barrel Bombs}{2d10 (small) or 4d10 (large) bludgeoning}
Considered crude by some hunters, the barrel bomb still remains a staple in the hunter's arsenal since the invention of gunpowder. Due to the unreliability of fuses, usually only the small variants~\smallicon{assets/ext/icons/barrel-bomb-yellow} include fuses, the large ones need to be triggered in other ways.

\imageheader{assets/ext/iconsL/horn-yellow}{Magic Horns}{}
These magically enchanted horns can be blown as an action and apply their effect to all friendly creatures within 100 feet who can hear it. The typical application is a healing horn~\smallicon{assets/ext/icons/horn-green}, but other variants are possible. Horns use up their magic after their first use and are just mundane horns afterwards.

\imageheader{assets/ext/iconsL/meat-orange}{Meat}{}
Choosing the location of the battle in your favour can make the difference between a successful hunt and a terrible failure. Baiting a monster with its preferred food is a popular way of making sure the fight happens on your terms. Once again, research is key to determine what food the monster prefers.

\imageheader{assets/ext/iconsL/book-blue}{Information}{}
From all these entries it should be obvious that knowing is half the battle. Observing the monster~\smallicon{assets/ext/icons/binoculars} or studying available texts about it is the way to find out which tricks work, and which tricks won't.


\section{Loot}

Besides knowing that the village is safe again, there is a variety of useful materials to be gained from the slain monsters themselves. Many parts of a monster can be put to good use. Here's what to watch out for:

\imageheader{assets/ext/iconsL/bone-white}{Bones}{}
Many monsters have particularly hard bones which are doubly useful in that they can be used to make weapons \emph{and} armor, as well as tools. Useful bones are usually obtained from beasts and leviathans---wyverns, especially flying and bird wyverns have fairly lightweight bones that are not as useful.

\imageheader{assets/ext/iconsL/scale-green}{Scales}{Or shells~\smallicon{assets/ext/icons/shell-green} or carapaces~\smallicon{assets/ext/icons/carapace-green}}
Wyverns in particular have hardened scales that are their primary protection. They are highly useful for armorcrafting. Scales come in all shapes and sizes, but usually, only the hardest and largest scales are useful, which do not always survive the fight undamaged. Other monsters have hardened exoskeletons, which are easier to obtain but usually make the monster very hard to kill. The rarest scales are so-called plates~\smallicon{assets/ext/icons/scale-yellow} or mantles~\smallicon{assets/ext/icons/mantle-yellow}. They make the best armor material, but are very difficult to carve. Scales, carapaces and plates are usually used to make medium or heavy armor.

\imageheader{assets/ext/iconsL/hide-pink}{Hides}{Or webbing~\smallicon{assets/ext/icons/webbing-pink}}
For crafting lighter armor, hard and heavy materials such as bones and scales are less useful than the lighter hides of beasts or webbings of winged monsters. Of these, hides are more commonly used for armor, as webbing is often more useful for the manufacture of magic items. However, webbing can also be used in armorcrafting if certain magical effects are desired. Hides and webbing are usually used to make light or medium armor.

\imageheader{assets/ext/iconsL/fang-cyan}{Fangs \& Claws}{}
Nearly every monster has these, but they are the first thing to be damaged in a fight and likely to be worn even before it has started. For crafting, only undamaged fangs and claws are useful. Besides crafting weapons, fangs and claws are frequently used as part of charms or decoration on armor.

\imageheader{assets/ext/iconsL/sac-yellow}{Sacs}{}
Sacs are glands that some monsters have developed to produce their special defenses, such as flammable liquid or venom. If they can be removed undamaged and before everything inside has been expended, they can be very useful. Even empty, some can be used as part of a magical enchantment.

\imageheader{assets/ext/iconsL/monster-jewel-red}{Gems}{}
The most sought after prize after defeating a wyvern is without question a wyvern gem. Unique to each species of wyvern, they are similar to pearls, forming in the digestive tract of the wyvern. Only very few wyvern form one, however, so these remain among the rarest trophies a hunter can obtain. If one manages, however, they are not only beautiful to look at but also extremely powerful magical catalysts and highly valuable.

\imageheader{assets/ext/iconsL/egg-white}{Eggs}{}
There is not much publically available information on the usefulness of the eggs of large wyverns. The gathering of wyvern eggs is a practice that is not only extremely dangerous, as the wyvern will protect them fiercely, but also forbidden by the Guild. When the Guild issues a quest to hunt a monster, that contract does not extend to the monster's offspring. In keeping with the Guild's tenet of \emph{life as a community}, the gathering of large wyvern eggs is considered poaching (no pun intended).


\section{Other Materials}
In addition to monster materials, other, naturally occuring items may be used in the creation of weapons, armor and other items as well as in alchemy and cooking.

\imageheader{assets/ext/iconsL/herb-green}{Herbs}{And mushrooms~\smallicon{assets/ext/icons/mushroom-cyan}}
There is a huge variety of useful herbs and mushrooms, if one knows where to find them. Many are seasonal, hard to find or grow only in extreme conditions. A skilled herbalist can make a comfortable living simply because some herbs are quite valuable owing to the difficulty of obtaining them.

\imageheader{assets/ext/iconsL/fish-cyan}{Fish}{}
While not typically useful for crafting or alchemy, fish are a common source of food and some have other, surprising usefulness, such as the whetfish~\smallicon{assets/ext/icons/fish-yellow} whose rough scales can be used to sharpen weapons.

\imageheader{assets/ext/iconsL/ore-orange}{Ores}{}
The people of the ancient civilisation were the true masters of working different types of ore. While much of their handiwork has been destroyed, what remains shows a degree of craftsmanship that today's society can only hope to match. Nevertheless, the field of metallurgy is advancing rapidly and new methods of working and extracting metal are being found nearly every day, perhaps even some that not even the ancients thought of. The center of innovation of these advances are mining towns and knowledge spreads only slowly outwards from these places, so not every village blacksmith will know how to work the more exotic metals.

\hbBottomRightArt{.72}{0.82}{0.9}{assets/ext/yukomo-village}

\begin{hbFancyWideTable}[b]{Food Effects}{LLLY}
                          & \textbf{Food Name} & \textbf{Buff Name} & \textbf{Effect}\\
\tableicon{meat-orange}   &                    & Attribute Booster  & Choose an attribute. You may roll a single check of this attribute with advantage. Afterwards, the effect ends.\\
\tableicon{fish-cyan}     &                    & Defense Up         & Whenever you take damage from an attack, there is a 10\% chance you take half damage instead. After this happens, the effect ends.\\
\tableicon{meat-red}      &                    & Attack Up          & When you hit with an attack, you may choose to deal an additional 1d6 damage. When you have used this effect twice, it ends.\\
\tableicon{potion-white}  &                    & Bombardier         & Small barrel bombs you use deal an additional 1d10 damage. Large barrel bombs you use deal an additional 2d10 damage.\\
\tableicon{sac-blue}      &                    & Carver        & You receive an additional carve from a single large monster.\\
\tableicon{potion-white}  &                    & Kickboxer          & Your unarmed attacks count as magical for the purposes of overcoming damage reduction and deal 1d8 bludgeoning damage.\\
\tableicon{herb-green}    &                    & Supercat           & For the purposes of lifting and pushing, your strength score is 20.\\
\tableicon{fish-orange}   & Glutton Filet      & Felyne Feet        & You have advantage on strength saving throws and you cannot be knocked prone as a result of a failed dexterity saving throw.\\
\end{hbFancyWideTable}

\newpage
\section{Food}

While many hunters have different rituals on what to do before embarking on a hunt, it is well known that a good meal before a hunt is very important. So important even, that a good cook is considered indispensable for any hunting caravan or expedition. A skilled cook can prepare a variety of foods appropriate to the situation to lift spirits and the hunters the strength they need to keep up the hunt. Even better if the cook is a hunter themself, so the party can continue to be supplied with good food in the field.

In addition to being nutritious, meals provide small, temporary effects. Any creature can only benefit from one meal effect between long rests. Eating a second meal before taking a long rest will not provide any bonus. Meal effects last until they are depleted (for example, temporary hit points are used up), are used up (for limited use effects) or the user takes a long rest, though they last for 24 hours at most. Not just any food will do to get an effect, though! Only food prepared by a skilled cook and made with the right ingredients will count for these effects.

The effects and foods given here are not the only ones, and you are encouraged to invent your own!

\newpage
\hbTopRightArt{1.06}{1.0}{1.0}{assets/ext/kittycook}
\null

\clearpage
\section{Special Weapons and Armor}
So you've done it, you've slain your first Rathalos and are probably not feeling quite good about yourself right now, but are also wondering what to do with the scales, bones, claws, wings, spikes and other parts of the beast that you've carved off for yourself. You may also have found yourself in the posession of some rare ore or another and are wondering what it could possibly be good for.

The thing about hunting monsters is: There is always a bigger fish. No matter if you've just bested Rathalos or even the mighty Rajang, there is always a tougher monster out there. And to face it, you will have to come with the proper equipment. The usual thing to do is to make it out of the bodyparts of the monster you've just slain. But weapons and armor from monster parts or rare ores does more than offer protection, it also allows your gear to be imbued with a variety of magical effects.

\imageheader{assets/ext/iconsL/helm-blue}{Velociprey Mask}{Made from \tableicon{carapace-blue}\,Velocidrome Head\\Requires Attunement}
\textit{A mask made to resemble the face of a Velociprey. Feels as if it has a soul.}

While wearing to this terrifying mask, you have proficiency with the intimidation skill. You have disadvantage on all charisma checks that aren't intimidation. In addition, when you take the attack action, you may perform a leaping attack and jump up to 20 feet before performing the first attack. This does not count against your movement allowance for the turn. Once you have used this ability, you may not use it again until you finish or short or long rest.

\newpage
\null

\begin{hbFancyWideTable}[b]{Weapon and Armor Enchantments}{LY}
\hiderowcolors
\textbf{Armor and Jewelry} & \textbf{Effect}\\
\showrowcolors
Biology & Prevents you from getting dirty from thrown poop or dung. Improves efficacy of dung bombs.\\
Carving Pro & Grants one additional carve if you dealt damage to the monster at most one minute before it died.\\
Constitution & You have advantage on saving throws made to resist fatigue.\\
Commander & You can take an action to inspire fallen comrades. Nearby allies who have 0 hit points regain one hit point. Recharges on a long rest.\\
Defense Up & Increases your armor class by 1, 2 or 3, depending on the level of this ability.\\
Divine Blessing & Whenever you would take damage, there is a 10\% chance that you take half damage instead.\\
Earplugs & Makes you immune to the effects of a monster's roar.\\
Elemental Resistance & Grants resistance to a particular damage type.\\
Evade Distance & Whenever you succeed at a dexterity saving throw, you may move up to 10 feet without triggering attacks of opportunity.\\
Focus & You have advantage on saving throws made to maintain concentration.\\
Guts & If you would be reduced to 0 hit points, you are reduced to 1 instead. Recharges on a long rest.\\
Health Up & Increases your hit point maximum by 1 for each hit dice you have.\\
Maestro & Whenever you use a healing horn~\tableicon{horn-green}, you may add your charisma bonus to the amount healed.\\
Meat Lover & You can eat raw meat as though it were cooked.\\
Mycology & You are able to eat any kind of mushroom for a variety of special effects. Use the food table for inspiration.\\
Mind's Eye & Reduce the AC when attacking specific parts of monsters, to a minimum of the monster's AC+1.\\
Mounting Master & You have advantage on checks made to begin or maintain a grapple.\\
Potential & You have advantage on dexterity saving throws while you have less than 40\% of your hit point maximum remaining.\\
Psychic & You know the locations of any large monsters within 1 mile of your location.\\
Recovery Up & Whenever you restore hit points, you restore an additional 1d6 hit points.\\
Rock Steady & You have advantage on strength saving throws.\\
Speed Eating & Allows you to consume a potion or food as a bonus action.\\
Sprinter & You may take the dash action as a bonus action.\\
\hiderowcolors
& \global\rownum=0\relax\\
\textbf{Weapon and Shield} & \textbf{Effect}\\
\showrowcolors
Adrenaline & Attacks made with this weapon deal an additional 2d6 damage while you have less than 40\% of your hit point maximum remaining.\\
Attack Up & Grants a +1, +2 or +3 bonus to attack and damage rolls made with this weapon, depending on the level of this ability.\\
Challenger & Attacks made with this weapon deal an additional 2d6 damage while the target is enraged.\\
Critical Draw & The first attack in each combat deals an additional die of damage.\\
Critical Eye & Increases the critical range for this weapon by 1, 2 or 3, depending on the level of this ability.\\
Guard Up & (Shield only) Your armor class increases by 1.\\
Precision & (Ranged only) The range of this weapon is doubled.\\
Tenderizer & Attacks made with this weapon against a specific part of a monster deal an additional 1d6 damage.
\end{hbFancyWideTable}

\newpage
\section{Weapon and Armor Enchantments}
In addition to the specific weapons and armor described above, a number of generic enchantments are available, which can be active on equipment of different types and from different monsters. You can use this list of enchantments for inspiration on designing your own equipment that you want to craft.

Generally speaking, gear made from metals will be receptive to enchantment by a craftsman who is skilled in such magic. Equipment made from monster parts will carry the innate properties of the monster from which it came. In either case, it means that not every enchantment will necessarily be available to you whenever you want it. Speak to your DM about what you can craft out of the items you have and the craftsmen who are at hand, as well as what materials are required.

\begin{hbFancyWideTable}[p]{Item Table}{LYR}
\showrowcolors
                                 \textbf{Item} & \textbf{Description} & \textbf{Value}\\
\tableicon{potion-green}         Potion & Restores a small amount of Health (2d4+2). & 66z\\
\tableicon{potion-green}         Mega Potion & Restores a moderate amount of Health (4d4+4). & 165z\\
\tableicon{potion-cyan}          Nutrients & Very slightly increases your maximum Health level (2d4+2). & 760z\\
\tableicon{potion-cyan}          Mega Nutrients & Slightly increases your maximum Health level (4d4+4). & 920z\\
\tableicon{potion-blue}          Antidote & Removes all traces of poison from your system. & 60z\\
\tableicon{potion-yellow}        Immunizer & Boosts your natural ability to heal (restore two hit dice). & 923z\\
\tableicon{potion-yellow}        Dash Juice & Lets you run without tiring for a short period of time (2 hours). & 293z\\
\tableicon{potion-yellow}        Mega Dash Juice & Lets you run without tiring for longer than regular Dash Juice does. & 1,028z\\
\tableicon{potion-red}           Demondrug & Boosts your Attack by filling you with\hbNone guess what?\hbNone demonic strength (STR=21 for 1 hour) & 666z\\
\tableicon{potion-red}           Mega Demondrug & Boosts your Attack even more than a regular Demondrug (STR=25 for 1 hour). & 2,666z\\
\tableicon{sac-red}              Might Pill & Temporarily endows you with the strength of a mighty god. Potent! (STR=30 for 1 round) & 2,666z\\
\tableicon{potion-orange}        Armorskin & Boosts your Defense by turning your skin as hard as rock. (+1AC for 1 hour) & 578z\\
\tableicon{potion-orange}        Mega Armorskin & Boosts your Defense even more than a regular Armorskin. (+2AC for 1 hour) & 2,696z\\
\tableicon{sac-orange}           Adamant Pill & Temporarily makes your skin as hard as adamant. Potent! (Immune to damage for 1 round) & 2,696z\\
\tableicon{potion-white}         Cool Drink & Provides temporary relief from extreme heat. (1 day) & 300z\\
\tableicon{potion-red}           Hot Drink & Provides temporary relief from extreme cold. (1 day) & 250z\\
\tableicon{potion-cyan}          Cleanser & Immediately removes any ice or webbing on your body. Clean as a whistle! & 140z\\
\tableicon{smoke-cyan}           Deodorant & An item that cures Stench. Releases a puff of deodorizing smoke when thrown down. & 80z\\
\tableicon{potion-orange}        Psychoserum & Temporarily sharpens your sixth sense and attunes you to the ways of monsters. & 300z\\
\tableicon{sac-white}            Herbal Medicine & Removes all traces of poison and restores a slight amount of Health. & 250z\\
\tableicon{sac-yellow}           Max Potion & Fully restores Health. & 2,138z\\
\tableicon{sac-red}              Ancient Potion & Fully restores Health and cures all diseases and afflictions. & 3,454z\\
\tableicon{sac-grey}             Catalyst & Works with other materials to enhance their effects. Cannot be used by itself. & 480z\\
\tableicon{sac-red}              Gunpowder & A dangerous substance that explodes when struck or heated. & 90z\\
\tableicon{sac-white}            Lifecrystals & Mysterious crystals long worshipped as a source of life. & 592z\\
\tableicon{sac-white}            Lifepowder & Medicine made from compounded Lifecrystals. Heals those within range with a single touch. & 6,300z\\
\tableicon{sac-teal}             Dust of Life & A medicine with strong curative properties. Studied in secret, it is said to restore youth. & \hbNone\\
\tableicon{sac-pink}             Disposable Earplugs & One-use-only earplugs that negate the effects of all large monsters' roars. & 400z\\
\tableicon{potion-red}           Tranquilizer & Medicine that works as an anesthetic when combined. & 150z\\
\tableicon{knife-white}          Throwing Knife & A standard throwing knife. Flies straight and true. & 2z\\
\tableicon{monster-jewel-yellow} Flash Bomb & Flashes brightly on impact. Toss this right under a monster's nose to blind it. & 572z\\
\tableicon{monster-jewel-grey}   Sonic Bomb & A grenade-like item that emits a high-frequency blast of sound on detonation. & 450z\\
\tableicon{dung}                 Dung Bomb & When thrown, releases a strong odor which certain monsters find repulsive. & 120z\\
\tableicon{smoke-white}          Smoke Bomb & Creates a large cloud of smoke wherever it lands. & 100z\\
\tableicon{smoke-purple}         Poison Smoke Bomb & Contains a toxic chemical. Also popular as a household bug bomb. & 600z\\
\tableicon{smoke-green}          Farcaster & Instantly transports you back to base camp. & 150z\\
\tableicon{smoke-yellow}         Portable Steam Bomb & Yukumo hot spring steam that cures all abnormal statuses. Everybody jump in! & \hbNone\\
\tableicon{barrel-yellow}        Small Barrel & A small, empty barrel. & 80z\\
\tableicon{barrel-red}           Large Barrel & A large, empty barrel. & 210z\\
\tableicon{barrel-bomb-yellow}   Small Barrel Bomb & A small time bomb. & 156z\\
\tableicon{barrel-bomb-red}      Large Barrel Bomb & Powerful bomb triggered by external physical impact. & 518z\\
\tableicon{horn-green}           Health Horn & Restores Health to those who hear it. May break when used. & 1,660z\\
\hiderowcolors
\multicolumn{3}{>{\hsize=3\hsize}L}{Value is buying price, selling price is $\frac{1}{10}$th} \\
\multicolumn{3}{>{\hsize=3\hsize}L}{\textit{Note: These are the prices from the video game right now. For \DND{}, these are probably all over the place.}}
\end{hbFancyWideTable}

\begin{hbFancyWideTable}[p]{Natural Resources}{LYR}
\showrowcolors
                           \textbf{Item} & \textbf{Description} & \textbf{Value}\\
\tableicon{webbing-orange} Honey & Sweet, golden honey. High in nutritional value. & 45z\\
\tableicon{herb-green}     Herb & Restores a very small amount of Health. Not very flavorful. & 2z\\
\tableicon{herb-blue}      Antidote Herb & A plant used in antidotes. It sometimes cures poison when eaten by itself. & 2z\\
\tableicon{herb-red}       Fire Herb & A wondrous, flammable plant. & 4z\\
\tableicon{herb-green}     Ivy & A lightweight and extremely strong plant. & 8z\\
\tableicon{herb-cyan}      Sleep Herb & A plant containing sleeping agents. & 5z\\
\tableicon{herb-white}     Sap Plant & A plant with leaves coated in sticky sap. Difficult to remove once attached. & 2z\\
\tableicon{herb-yellow}    Felvine & Melynxes' favorite food\hbNone they just have to steal it. & 1z\\
\tableicon{herb-pink}      Gloamgrass Root & Possesses antidotal agents, but it will not work by itself. & 10z\\
\tableicon{herb-teal}      Gloamgrass Bud & A flower that dislikes light and blooms in darkness. May possess healing agents. & 350z\\
\tableicon{herb-red}       Hot Pepper & Blazingly spicy. Sure, it'll warm you up, but it's too hot to handle on its own. & 4z\\
\tableicon{seed-pink}      Frozen Berry & A completely frozen berry. Can provide relief from heat and recover Stamina. & 54z\\
\tableicon{seed-white}     Cathangea Seed & A seed used in a Wyverian healing agent. Highly sought after by merchants. & 100z\\
\tableicon{herb-pink}      Cathangea & A flower used in a Wyverian healing agent. Highly sought after by merchants. & 200z\\
\tableicon{sac-red}        Cathangea Elixir & A popular Wyverian healing agent that highly sought after by merchants. & 300z\\
\tableicon{mushroom-blue}  Blue Mushroom & Rare mushroom with a power-enhancing effect. & 2z\\
\tableicon{mushroom-red}   Nitroshroom & A hard-to-find mushroom with the power to generate blazing heat. & 6z\\
\tableicon{mushroom-yellow} Parashroom & A mushroom that induces paralysis. & 15z\\
\tableicon{mushroom-purple} Toadstool & Logic says not to eat this poisonous fungus, but don't write it off completely. & 8z\\
\tableicon{mushroom-purple} Exciteshroom & This one has a very strange aroma... But is it edible? Only one way to find out. & 18z\\
\tableicon{mushroom-cyan}  Mopeshroom & A mushroom that contains a Stamina-stealing constituent. & 9z\\
\tableicon{mushroom-red}   Dragon Toadstool & A dangerous fungus said to draw the life out of people. Beware. & 40z\\
\tableicon{seed-pink}      Paintberry & A berry with a shocking hue and an overwhelming scent. Look out: It WILL stain! & 6z\\
\tableicon{seed-red}       Might Seed & Temporarily raises your Attack when ingested by improving energy flow. & 140z\\
\tableicon{seed-orange}    Armor Seed & Temporarily raises your Defense when ingested by hardening tissue. & 110z\\
\tableicon{seed-blue}      Nulberry & A mysterious berry that cures various Blights and also protects from unknown viruses. & 120z\\
\tableicon{seed-red}       Dragonfell Berry & A mysterious berry, rumored since ancient times to be loathed by dragons. & 78z\\
\tableicon{seed-grey}      Scatternut & A nut that violently ejects its contents when struck. & 4z\\
\tableicon{seed-grey}      Needleberry & A berry covered with needle- like thorns. Used as a material for ammo. & 1z\\
\tableicon{seed-cyan}      Latchberry & A berry with spiral grooves. It spins when it falls to firmly embed itself in soil. & 4z\\
\tableicon{seed-grey}      Bomberry & A berry that explodes when struck. & 12z\\
\tableicon{seed-orange}    Bumblepumpkin & A trendy orange pumpkin. If left undisturbed, it can grow to gigantic proportions. & 20z\\
\tableicon{ore-grey}       Iron Ore & Ore that can be smelted into metal and used for many different purposes. & 60z\\
\tableicon{ore-white}      Earth Crystal & Crystallized microbes which are prized as an abrasive when forging weapons. & 80z\\
\tableicon{ore-blue}       Machalite Ore & An ore that yields better metals than Iron Ore. Used to make Machalite. & 160z\\
\tableicon{ore-green}      Dragonite Ore & An ore that yields metal superior to that of Machalite. Rare and valuable. & 480z\\
\tableicon{ore-purple}     Carbalite Ore & An ore still being researched. It yields even better metal than Dragonite. & 680z\\
\tableicon{ore-red}        Eltalite Ore & A promising new ore that produces higher quality metal than Carbalite Ore. & 1,280z\\
\tableicon{ore-grey}       Lightcrystal & An extremely hard substance with a faint glow. Sometimes used for crafting tools. & 1,150z\\
\tableicon{ore-white}      Novacrystal & A brightly glowing crystal. Extremely hard, it's also used to make armory tools. & 2,440z\\
\tableicon{ore-cyan}       Ice Crystal & Ice that will not melt at room temperature. Can be used to create denser ore. & 60z\\
\tableicon{ore-red}        Firestone & Combusts even at room temperature. Its high heat can be used to fuse materials. & 860z\\
\tableicon{ore-pink}       Fucium Ore & Ore composed of a mysterious metal; capable of fusing nearly any two materials together. & 1,020z\\
\tableicon{ore-white}      Meldspar Ore & An ore composed of a durable, flexible metal that works well as a combining agent. & 1,920z\\
\tableicon{ore-teal}       Gossamite Ore & An ore that yields a strong but shockingly light metal. Still a scientific mystery. & 720z\\
\tableicon{ore-white}      Heavenly Crystal & A valuable crystal with few impurities that is used for polishing hard weapons. & 80z\\
\tableicon{ore-white}      Meteor Crystal & A precious abrasive capable of polishing hard metals to unlock their true potential. & 1,280z\\
\tableicon{ore-orange}     Firecell Stone & A magma-like deposit that only trained hands can properly work with. & 1,720z\\
\tableicon{ore-red}        Allfire Stone & An ore that blazes with an all-consuming flame. Requires considerable skill to mine. & 5,160z\\
\tableicon{ore-cyan}       Yoldspar Stone & A heavenly ore found on the highest mountaintops. Hard to forge, but versatile. & 1,440z\\
\tableicon{ore-purple}     Cosmicite Ore & A precious ore that gains great strength when alloyed with other metals. & 4,320z\\
\tableicon{ore-cyan}       Purecrystal & A crystal that's 100\% pure. Hard enough to be used in workshop tools. & 7,320z\\
\hiderowcolors
\multicolumn{3}{>{\hsize=3\hsize}L}{Value is selling price, buying may not be available.}
\end{hbFancyWideTable}
%\clearpage

\begin{hbFancyWideTable}[p]{Natural Resources (cont.)}{LYR}
\showrowcolors
                           \textbf{Item} & \textbf{Description} & \textbf{Value}\\
\tableicon{ore-white}      Armor Stone & Reacts uniquely to heat. Combine with other materials to create equipment coating. & 150z\\
\tableicon{fish-yellow}    Whetfish & A fish with a dorsal fin hard enough to be used to sharpen weapons. & 45z\\
\tableicon{fish-orange}    Sushifish & A fatty, delicious fish that restores a small amount of Health when consumed. & 45z\\
\tableicon{fish-cyan}      Sleepyfish & A fish that contains sleeping agents. & 45z\\
\tableicon{fish-grey}      Pin Tuna & A fish with jaws covered in tiny, needle-like spikes. & 50z\\
\tableicon{fish-blue}      Speartuna & A giant fish that's both rare and valuable. It looks like it could be used for something\ldots & 1,000z\\
\tableicon{fish-grey}      Popfish & A fish that literally pops when it dies. & 15z\\
\tableicon{fish-grey}      Scatterfish & A fish that splits into little pieces when it dies. & 150z\\
\tableicon{fish-green}     Burst Arrowana & A fish that bursts open when it dies. & 45z\\
\tableicon{fish-purple}    Bomb Arrowana & A fish that explodes when it dies. & 135z\\
\tableicon{fish-orange}    Glutton Tuna & A fish that eats anything\hbNone allowing you to sometimes get items from its stomach. & 13z\\
\tableicon{fish-orange}    Gastronome Tuna & An enormous Glutton Tuna. Will eat anything, no matter the size. & 23z\\
\tableicon{fish-yellow}    Small Goldenfish & A small, rare, gold fish. Commands a high price. & 500z\\
\tableicon{fish-green}     Wanchovy & A fish that can induce exhaustion. & 48z\\
\tableicon{fish-grey}      Ancient Fish & A precious fish; almost every part can be used as material. & 1,500z\\
\tableicon{fish-cyan}      Blue Cutthroat & A fish that emits blue light. Its sharp body will cut you to pieces if you're not careful. & 100z\\
\tableicon{fish-grey}      Coinperch S & A small aquarium fish that village children catch and then sell for pocket change. & 150z\\
\tableicon{fish-yellow}    Coinperch M & A fish that some families keep for good luck, but most sell for the extra money. & 500z\\
\tableicon{fish-yellow}    Coinperch L & A fish said to move whoever catches it up in the world. Sells for serious moolah. & 2,500z\\
\tableicon{fish-red}       Dundorma Tuna & A huge and valuable fish from Dundorma. Someone must be willing to trade for it. & 20z\\
\tableicon{fish-purple}    Cathangeafish Fry & A young fish prized by Wyverians. Makes a great gift for their merchants. & 80z\\
\tableicon{fish-purple}    Cathangeafish & A fish prized by Wyverians. Makes a great gift for their merchants. & 120z\\
\tableicon{bug-grey}       Insect Husk & The remains of a dead insect. & 1z\\
\tableicon{bug-red}        Stinkhopper & A grasshopper with a foul odor. Make fishing lures with it; its scent drives explosive fish wild. & 6z\\
\tableicon{bug-orange}     Snakebee Larva & An unusual larva to make fishing lures with; its scent drives the best fish wild. & 30z\\
\tableicon{bug-white}      Godbug & An insect said to live for a thousand years. & 210z\\
\tableicon{bug-blue}       Bitterbug & A sharp-tasting bug with innate healing abilities. Eat one for a 50\% chance of curing poison. & 2z\\
\tableicon{bug-yellow}     Flashbug & An insect that emits a powerful flash when it dies. & 48z\\
\tableicon{bug-yellow}     Thunderbug & An insect that emits electricity when struck. Has many applications. & 150z\\
\tableicon{bug-pink}       Glueglopper & A cricket-like insect that disgorges an extremely powerful adhesive. & 120z\\
\tableicon{bug-green}      Killer Beetle & An insect with dazzling wings. Its hard shell is often used as an abrasive. & 200z\\
\tableicon{bug-red}        Hercudrome & Said to be the world's strongest bug. Used in forging and other things. & 540z\\
\tableicon{bug-blue}       Fulgurbug & Thunderbugs which have been stimulated by Zinogre. They emit a pale blue light. & 350z\\
\tableicon{bug-yellow}     Rare Scarab & A rare and valuable insect that greatly assists in the bonding of metals. & 1000z\\
\tableicon{bug-orange}     Emperor Hopper & A grasshopper with an exceptionally strong shell. Prized as a crafting material. & 1200z\\
\tableicon{bug-white}      Flutterfly & A butterfly with shimmering wings. So rarely seen in the wild many believe it to be a myth. & 1800z\\
\tableicon{bug-yellow}     Great Hornfly & An insect with a giant horn, massive shell, and wings of butterfly-like beauty. & 60z\\
\tableicon{bug-purple}     Elder Butterfly Beetle & A very strange insect with a beetle's forked jaws and the vivid wings of a butterfly. & 500z\\
\tableicon{bug-grey}       Butterfly Beetle & A strange insect with the large jaws of a beetle and the delicate wings of a butterfly. & 220z\\
\tableicon{bug-purple}     Great Elder Hornfly & An insect with two giant horns, a massive shell, and a butterfly's true beauty. & 330z\\
\tableicon{bone-yellow}    Monster Bone S & A very useful material; indispensable for both hunting and daily life. & 10z\\
\tableicon{bone-yellow}    Monster Bone M & A nicely-sized bone. Strong yet flexible, it's as reliable as steel and wood. & 210z\\
\tableicon{bone-yellow}    Monster Bone L & Wyvern bone needed to craft larger items. The right item will fuse this to a shell. & 440z\\
\tableicon{bone-yellow}    Mystery Bone & A weathered bone. Not sturdy enough to use for crafting. & 1z\\
\tableicon{bone-yellow}    Unknown Skull & Animal skull. So worn and weathered, it's unidentifiable. & 120z\\
\tableicon{bone-white}     Brute Bone & The stout bone of a brutish monster. An easy-to-work material. & 20z\\
\tableicon{bone-white}     Jumbo Bone & A large, thick bone. Often cut apart in workshops for use in crafting. & 150z\\
\tableicon{bone-white}     Dragonbone Relic & A curiously unfossilized bone from a dragon that once roamed these lands. & 500z\\
\tableicon{claw-white}     Wyvern Fang & Used in Bowgun bullet points. Gunpowder shatters these, generating shrapnel. & 6z\\
\tableicon{claw-white}     Wyvern Claw & Used in Bowgun bullet points. Gunpowder triggers a blast with a wide radius. & 18z\\
\hiderowcolors
\multicolumn{3}{>{\hsize=3\hsize}L}{Value is selling price, buying may not be available.}
\end{hbFancyWideTable}

\chapter{The Wilderness}
\hbWideBottomArtFirstPage{1.98}{.892}{assets/ext/egg-hunt}
The wilderness comes in many different kinds, from scorching deserts to humid jungles and freezing tundras. Monsters thrive in every biome, from lush plains to extremes like volcanoes or the depths of the ocean. What follows is a description of the basic terrain types, corresponding to the ranger's favored terrain types: arctic, coast, desert, forest, grassland, mountain, swamp, underground and as a new addition: underwater.

\subsubsection{Foraging}
A hunting party travelling through the wilderness will often have to forage for food or gather materials for crafting. Rather than treat this as a bookkeeping exercise or a gruelling survival scenario, I prefer to treat this as an opportunity for hijinks and adventure. Nevertheless, the rules for wilderness survival from the \DMG{} apply: Each character needs one pound of food and one gallon of water per day to stay healthy.

You can forage for food while you travel, so long as you are travelling at slow or medium pace. A foraging character makes a Wisdom (Survival) check, with the DC determined by the abundance of food and water in the region.

\begin{hbNarrowTable}{Foraging DCs}{YC}
\textbf{Food and Water Availability} & \textbf{DC}\\
Abundant food and water sources & 10\\
Limited food and water sources & 15\\
Very little, if any, food and water sources & 20
\end{hbNarrowTable}

If multiple characters forage, each character makes a seperate check. A foraging character finds nothing on a failed check (and may run into trouble, such as a random encounter). On a successful check, roll 1d6+the character's Wisdom modifier to determine how much food (in pounds) the character finds, then repeat the roll for water (in gallons).

\subsubsection{Hunting}
In order to hunt for food, a character must hunt for at least 4 hours, which cannot be done while travelling. Roll on the following chart to see if any prey is found. If you have proficiency with the Survival skill, this roll is with advantage. Then refer to the section about the terrain type to roll what you found.

\begin{hbWideTable}[b]{Random Weather Table}{RYYYL}
\textbf{d100} & \textbf{Weather} & \textbf{Cold Climate} & \textbf{Temperate Climate} & \textbf{Desert}\\
01-70 & Normal Weather & Cold, calm & Normal for season & Hot, calm\\
71-80 & Abnormal Weather & Heat Wave (71-73) or cold snap (74-80) & Heat wave (71-75) or cold snap (76-80) & Hot, windy\\
81-90 & Inclement Weather & Precipitation (snow) & Precipitation (normal for season) & Hot, windy\\
91-99 & Storm & Snowstorm & Thunderstorm, snowstorm & Duststorm\\
100 & Powerful Storm & Blizzard & Windstorm, blizzard, hurricane, tornado & Downpour
\end{hbWideTable}

\begin{hbNarrowTable}{Hunting DCs}{YC}
\textbf{Terrain Type} & \textbf{DC}\\
Forest & 10\\
Grassland & 10\\
Coast & 12\\
Underwater & 12\\
Swamp & 13\\
Mountain & 15\\
Underground & 18\\
Desert & 20\\
Arctic & 20
\end{hbNarrowTable}

\hbWideBottomArtFirstPageFix

\subsubsection{Mining}
In order to mine ores, a character must mine for at least 4 hours, which cannot be done while travelling. Roll on the following chart to see if any ore is found. If you have proficiency with the Survival skill, this roll is with advantage. Then refer to the section about the terrain type to roll what you found. If you are proficient in the Survival skill, you add your proficiency modifier to the roll.

\begin{hbNarrowTable}{Mining DCs}{YC}
\textbf{Terrain Type} & \textbf{DC}\\
Underground & 10\\
Mountain & 10\\
Desert & 12\\
Coast & 13\\
Arctic & 13\\
Grassland & 15\\
Forest & 15\\
Underwater & 18\\
Swamp & 20\\
\end{hbNarrowTable}

\subsubsection{Fishing}
In order to catch fish, you must spend at least 4 hours fishing, which cannot be done while travelling. If you have fishing equipment, make a DC\,12 Wisdom (Survival) check. If you have excellent bait, you have advantage on this check. On a success, consult the table below to see what you caught. You cannot fish underwater, use the rules for hunting instead. Not every fish is good to eat!

\begin{hbNarrowTable}{Random Fish Table}{RYC}
\textbf{d20} & \textbf{Fish Type} & \textbf{Lbs. of meat}\\
1-2 & \tableicon{fish-grey} Scatterfish & 1d4\\
3 & \tableicon{fish-purple} Bomb Arrowana & 1d4\\
4 & \tableicon{fish-green} Burst Arrowana & 1d4\\
5-8 & \tableicon{fish-grey} Pin Tuna & 2d4\\
9-11 & \tableicon{fish-orange} Glutton Tuna & 4d4\\
12 & \tableicon{fish-orange} Gastronome Tuna & 6d4\\
13-14 & \tableicon{fish-cyan} Sleepyfish & 1d4\\
15 & \tableicon{fish-grey} Ancient Fish & 1d4\\
16-18 & \tableicon{fish-orange} Sushifish & 3d4\\
19 & \tableicon{fish-green} Wanchovy & 2d4\\
20 & \tableicon{fish-yellow} Whetfish & 1d4
\end{hbNarrowTable}

Random weather, random ores

\section{Arctic}
\lipsum[1]

\section{Coast}
\lipsum[1]

\section{Desert}
\lipsum[1]

\section{Forest}
The forest biome covers temperate and arctic forests, such as the ones found all over Goldora and Fonlon, as well as jungles, such as the ones in Aya. In the forests of the world, the battle for survival is at its fiercest, as neither raw strength or great speed will be enough to ensure survival. Despite the dense monster population of forests, many roads are built to lead through them, as the trees can provide cover from aerial wyverns.

Many misteries are hidden deep within impenetrable forests, both natural and artificial, both ancient and new.

\begin{hbNarrowTable}{Forest Hunting}{RYC}
\textbf{d10} & \textbf{Monster} & \textbf{Lbs. of Meat}\\
1-2 & \tableicon{altarothS} Altaroth & 1d4\\
3-4 & \tableicon{bullfangoS} Bullfango$\times$1d2 & 20+1d8\\
5-6 & \tableicon{kelbiS} Kelbi & 15+1d8\\
7-8 & \tableicon{epiothS} Epioth & 3d4$\times$10\\
9-10 & \tableicon{slagtothS} Slagtoth & 4d4$\times$20
\end{hbNarrowTable}

\begin{hbNarrowTable}{Jungle Hunting}{RYC}
\textbf{d10} & \textbf{Monster} & \textbf{Lbs. of Meat}\\
1-2 & \tableicon{altarothS} Altaroth & 1d4\\
3-4 & \tableicon{congaS} Conga & 20+1d10\\
5-6 & \tableicon{kelbiS} Kelbi & 15+1d8\\
7-8 & \tableicon{konchuS} Konchu & 3d4\\
9-10 & \tableicon{slagtothS} Slagtoth & 4d4$\times$20
\end{hbNarrowTable}

\begin{hbNarrowTable}{Forest and Jungle Mining}{RY}
\textbf{d20} & \textbf{Yield}\\
1-3 & \tableicon{ore-grey} Iron Ore\\
4-7 & \tableicon{ore-white} Earth Crystal\\
8-11 & \tableicon{ore-white} Heavenly Crystal\\
12-15 & \tableicon{ore-grey} Lightcrystal\\
16-19 & \tableicon{ore-white} Novacrystal\\
20+ & \tableicon{ore-cyan} Purecrystal
\end{hbNarrowTable}

\begin{hbNarrowTable}{Forest/Jungle Random Encounters}{RY}
\textbf{d100} & \textbf{Encounter}\\
 & \tableicon{bnahabraS} Bnahabra$\times$2d4\\
 & \tableicon{jaggiS} Jaggi/\tableicon{genpreyS} Genprey$\times$2d4\\
 & \tableicon{ludrothS} Ludroth/\tableicon{maccaoS} Maccao$\times$1d4+1\\
 & \tableicon{melynxS} Melynx/\smallicon{assets/ext/iconsL/shakalaka} Shakalaka\\
 & An exploration party attempting to map the forest.\\
 & \tableicon{great-jaggiS} Great Jaggi/\tableicon{gendromeS} Gendrome\\
 & \tableicon{bulldromeS} Bulldrome/\tableicon{kecha-wachaS} Kecha Wacha\\
 & \tableicon{royal-ludrothS} Royal Ludroth/\tableicon{great-maccaoS} Great Maccao\\
 & \tableicon{arzurosS} Arzuros/\tableicon{congalalaS} Congalala\\
 & Hunters searching for an unknown monster\\
 & \tableicon{qurupecoS} Qurupeco/\tableicon{gypcerosS} Gypceros\\
 & \tableicon{rathianS} Rathian/\tableicon{malfestioS} Malfestio\\
 & \tableicon{rathalosS} Rathalos\\
 & \tableicon{duramborosS} Duramboros/\tableicon{glavenusS} Glavenus\\
 & \tableicon{nargacugaS} Nargacuga/\tableicon{najaralaS} Najarala\\
 & \tableicon{zinogreS} Zinogre\\
 & \tableicon{deviljhoS} Deviljho/\tableicon{rajangS} Rajang\\
 & \smallicon{assets/ext/iconsL/yama-tsukami} Yama Tsukami/\tableicon{chameleosS} Chameleos
\end{hbNarrowTable}

\section{Grassland}
Grasslands exist in different climates, in the tropical savannas south of Arcolis, the temperate hills of Schrade or the cold tundras of Akra. In order to survive here, a monster must either be large enough to be hard to kill, or very fast, to escape predators, as there is barely any place to hide. There are not many roads leading through grasslands, as a caravan travelling through such an open area would be too exposed.

Some of the greatest marvels of the ancient civilisation can be found out on the plains. The monuments that they have left can sometimes be seen from a dozen miles away.

\begin{hbNarrowTable}{Grassland Hunting}{RYC}
\textbf{d10} & \textbf{Monster} & \textbf{Lbs. of Meat}\\
1-2 & \tableicon{aptonothS} Aptonoth & 4d4$\times$20\\
3-4 & \tableicon{bullfangoS} Bullfango$\times$1d2 & 20+1d8\\
5-7 & \tableicon{kelbiS} Kelbi$\times$1d4 & 15+1d8\\
8-10 & \tableicon{mosswineS} Mosswine$\times$1d4 & 15+1d12
\end{hbNarrowTable}

\begin{hbNarrowTable}{Grassland Mining}{RY}
\textbf{d20} & \textbf{Yield}\\
1-3 & \tableicon{ore-grey} Iron Ore\\
4-7 & \tableicon{ore-blue} Machalite Ore\\
8-11 & \tableicon{ore-green} Dragonite Ore\\
12-15 & \tableicon{ore-purple} Carbalite Ore\\
16-19 & \tableicon{ore-orange} Fucium Ore\\
20+ & \tableicon{ore-red} Eltalite Ore
\end{hbNarrowTable}

\begin{hbNarrowTable}{Grassland Random Encounters}{RY}
\textbf{d100} & \textbf{Encounter}\\
1-5 & \tableicon{vespoidS} Vespoid$\times$2d4\\
6-10 & A group of \tableicon{mosswineS}~Mosswine excitedly digging for something. When investigating, you find several \tableicon{mushroom-white}~Choice Mushrooms and a charm with the Mycology ability.\\
11-18 & You discover tracks. By following them or examining with your knowledge of monsters, you learn that they from a \tableicon{bulldromeS}~Bulldrome, which may attack.\\
19-25 & You discover a well-hidden ruin built by the ancient civilisation.\\
26-33 & \tableicon{velocipreyS}~Velociprey$\times$2d4.\\
34-43 & You come across several \tableicon{melynxS}~Melynx on the prowl.\\
44-53 & \tableicon{velocidromeS}~Velocidrome\\
54-60 & You are overtaken by a member of the \tableicon{felyneS}~Felyne mail, mounted on a \tableicon{gargwaS}~Gargwa\\
61-68 & \tableicon{yian-kut-kuS}~Yian Kut-Ku\\
69-75 & You come across another hunter party, who are on a quest of their own.\\
76-80 & 2d10 stampeding \tableicon{aptonothS}~Aptonoth\\
81-85 & \tableicon{rathianS}~Rathian\\
86-90 & Near a stream, a caravan has become stuck in the mud.\\
91-93 & You smell smoke. It is coming from a bushfire some distance away. There is a 50\% chance it was caused by a monster.\\
94-95 & \tableicon{deviljhoS}~Deviljho\\
96-98 & From some distance away, you spot the lumbering figure of \smallicon{assets/ext/iconsL/lao-shan-lung}~Lao-Shan Lung.\\
99-100 & You catch a glimpse of \tableicon{kushala-daoraS}~Kushala Daora, who is preceeded by a massive thunderstorm.\\
\end{hbNarrowTable}

\section{Mountain}
\lipsum[1]

\section{Swamp}
\lipsum[1]

\section{Underground}
\lipsum[1]

\section{Underwater}
\lipsum[1]
