%!TEX root = monsterhunter.tex

\newcommand{\monsterheader}[4]{%
\vspace{10pt plus 10pt minus 2pt}%
\noindent\begin{minipage}[c]{1cm}%
\includegraphics[width=\linewidth]{#1.png}%
\end{minipage}\hfill%
\begin{minipage}{\dimexpr -1em-1cm+\linewidth}%
\subsection*{#2}\index{#2}%
{\textit{Rating: } \raisebox{-2pt}[0pt]{#3} \textit{(#4)}\par}%
\end{minipage}\nopagebreak\\*[3pt]}

\newcommand{\sizecrowns}[2]{\hfill\mbox{\footnotesize\tableicon{crown-small}\,#1\,cm\quad\tableicon{crown-gold}\,#2\,cm\par}}

\renewcommand*{\hbPartCover}{assets/ext/lagiacrus}
\renewcommand*{\hbPartSubcover}{assets/ext/lagiacrus2}
\part{Monsters}

\chapter{Monster Listings}
The world of \MH{} is teeming with, you guessed it, monsters. The term "Monster" generally refers to any animal, even domesticated ones, with the possible exception of some very cute kittens, and poogies.

\paragraph{Creature Types.} The creature types of \DND{} are aberration, beast, celestial, construct, dragon, elemental, fey, fiend, giant, humanoid, monstrosity, ooze, plant and undead. Most of these are not applicable to \MH{}, so we are using different ones to balance out the preferred enemy of the ranger. The \MH{} creature types are lynian, herbivore, neopteron, temnoceran, bird wyvern, flying wyvern, piscine wyvern, carapaceon, amphibian, fanged beast, leviathan, snake wyvern, brute wyvern, fanged wyvern and elder dragon. That's too many and some of these categories have only a single entry.

Therefor, the creature types that we are using are: Beast, bird wyvern, flying wyvern, brute wyvern, leviathan and elder dragon. The following sections will list the monsters sorted by these categories.

\paragraph{But what about the undead?} Some classes, particularly the cleric, lose out on quite a few abilities if there are no undead. However, don't discard those features just yet, there might just be something that comes quite close enough for those abilities to work\ldots

\paragraph{Monster Ratings.} Officially, the monsters of \MH{} have star ratings from one star to six stars (with some exceptions like Alatreon having eight). Quests, however, are rated with one to ten stars and require a hunter rank of at least the amount of stars the quest has to undertake. \DND{} ranks monsters by challenge rating, where a monster of a certain challenge rating is supposed to be a challenge to a party of that level (this does \emph{not} work in practice, by the way). Here, I will rank the monsters with one through ten stars to differentiate the levels of \MH{} some more. However, this is not a precise science and as a DM, you are more than welcome to assign a different level to one of the monsters and have your party face it at an earlier or later time. Just like in \emph{Dragonball}, \emph{power levels are bullshit}.

\paragraph{I am a player, should I read these?} Talk to your DM. My thoughts on it: If you would like to discover or research the details on these monsters yourself, then don't read this chapter, or only the short descriptions (the cursive text under the monster heading). If you want your character to be the knowledgable sort and enjoy the increased agency that comes from knowing these details ahead of time, go ahead and read this chapter.

\paragraph{Sizes.} The sizes represent \emph{typical} monster size ranges. There may be individuals larger or smaller than the listed range, but the majority of monsters will fall within their listed size range. As a DM, you may consider scaling the size of the monster with the (rolled) hit point total, if you roll for the totals. That way, the players can find out through observation how strong the monster they are about to face is.

Without further ado, on to the monsters.

\begin{hbFancyWideTable}{Monster By Rating}{RYYYYY}
\textbf{Rating} & \textbf{Beasts} & \textbf{Brute Wyverns} & \textbf{Bird Wyverns} & \textbf{Flying Wyverns} & \textbf{Leviathans}\\
No Stars & Altaroth, Anteka, Bnahabra, Moofah & Aptonoth, Larinoth & Gargwa & \hbNone &  Fish\\
One Star & \hbNone & \hbNone & Jaggi, Velociprey & \hbNone & Delex, Ludroth\\
Two Stars & Arzuros & \hbNone & Velocidrome & \hbNone & \hbNone\\
Three Stars & \hbNone & \hbNone & \hbNone & \hbNone & Plesioth\\
Four Stars & \hbNone & \hbNone & \hbNone & Khezu, Rathian, Rathalos & \hbNone\\
Five Stars & \hbNone & Zinogre & \hbNone & \hbNone & \hbNone\\
Six Stars & \hbNone & \hbNone & \hbNone & Tigrex & \hbNone\\
Seven Stars & \hbNone & Deviljho & \hbNone & \hbNone & \hbNone\\
\end{hbFancyWideTable}

\let\svaddcontentsline\addcontentsline
\renewcommand\addcontentsline[3]{%
  \ifthenelse{\equal{#1}{lof}}{}%
  {\ifthenelse{\equal{#1}{lot}}{}{\svaddcontentsline{#1}{#2}{#3}}}}

\section{Beasts}
Beasts is a term referring to mammals of every kind, insects as well as some birds.

\monsterheader{assets/ext/icons/altaroth}{Altaroth}{\FiveStarOpen}{No Stars}
\textit{Insects that widely inhabit many areas. They absorb fruit, mushrooms, and honey, and then carry them back to their nest. Materials can thus be collected from their swollen abdomens, whose color is related to what they're carrying.}
\subsubsection{Physiology \& Ecology}
Altaroth are green ant-like insects growing up to 120cm long. Their appearance is characterised by a thin waist, a swollen, brightly colored abdomen and a large yellow crest which looks rather like a bonnet. Altaroth eat plants, mushrooms, nuts and sometimes roots, which they break down using their long mandibles and store in their abdomens, which can swell up to several times its normal size. When it does, the color of the material they have ingested is visible, giving the abdomen a varying bright coloring. In addition to their mandibles, they rely on a stinger for defense, as well as shooting corrosive formic acid from their stomach.

Altaroth occur in most regions and climates, they can even be found in arctic and desert areas, though there they keep to sheltered areas. There is some regional variety in the size and shape of the crest. In addition, there are two types of Altaroth, depending on their role within the hive: Soldier and worker. Soldiers have bigger crests and stronger mandibles but cannot store nutrients like workers can. However, even soldier Altaroth are hardly a threat to humanoids, as they are quite weak and their exoskeleton is considerably weaker than that of other insects.

\subsubsection{Behaviour}
Altaroth are usually found in groups of three to five, though they form large communities of up to a hundred individuals all living together in a huge nest. When encountered in the wild, a worker group is usually escorted by at least one soldier, though they will not attack if not disturbed. The distinction between the types is not believed to be a genetic difference, but a result of special food when young. This allows the hive to breed the type of Altaroth that is needed whenever it is required. It is also suspected that an Altaroth hive is somehow connected, with the individuals receiving orders remotely, perhaps from some kind of king or queen individual. However, there is no confirmed sighting of such a specimen.

Hunters and gatherers often find Altaroth useful because one can easily tell the existence and type of nearby resources by the color of their abdomens. Some herbalists and alchemists also attest special properties to the digested nutrients the Altaroth carry. Unprocessed, this substance is not edible.

\subsubsection{Carves}
Like with many insects, the best way to make sure that anything salvageable remains of a slain Altaroth is to use poison, as otherwise it might break and nothing useful remains.

\begin{hbNarrowTable}{Altaroth Carves}{RY}
\textbf{d100} & \textbf{Carve}\\
01-50 & \tableicon{sac-slime} Altaroth Stomach\\
51-75 & \tableicon{potion-cyan} Monster Fluid\\
76-100 & \tableicon{claw-slime} Altaroth Jaw
\end{hbNarrowTable}

\begin{hbNarrowTable}{Altaroth Abdomen Content}{RY}
\textbf{d100} & \textbf{Carve}\\
01-45 & \tableicon{mushroom-cyan} Ripened Mushroom\\
46-90 & \tableicon{potion-blue} Monster Broth\\
91-100 & \tableicon{webbing-orange} Honey
\end{hbNarrowTable}

\monsterheader{assets/ext/icons/anteka}{Anteka}{\FiveStarOpen}{No Stars}
\textit{Small herbivores known for their gentle demeanour as well as medicinal properties of their horns. Those wishing to harvest a horn should stun the Kelbi with blunt attacks and then sever it while the animal is still alive.}

\monsterheader{assets/ext/iconsL/arzuros}{Arzuros}{\FiveStar\FiveStar}{2 Stars}
\textit{Forest- and mountain-dwelling beasts found in humid regions. Though known more for fishing and standing upright to collect honey, their thick claws and heavy forearm plating allow them to deliver powerful blows to any aggressor.}\sizecrowns{506}{692}%{501}{854}
\subsubsection{Physiology \& Ecology}
Arzuros is a four-legged omnivorous beast with an ursine body structure. Particularly notable are its particularly strong and dextrous forelegs as well as the armor plating covering its head, upper back and arms.



\subsubsection{Behaviour}

\subsubsection{Carves}

\monsterheader{assets/ext/iconsL/bnahabra}{Bnahabra}{\FiveStarOpen}{No Stars}
\textit{Pervasive flying insects that attack invaders with paralyzing venom and lay eggs in carrion along with a fluid that hastens decomposition. It is best to kill them with poison so that their parts are left ripe for the carving.}
%\subsubsection{Physiology \& Ecology}
%Test
%\subsubsection{Behaviour}
%Test
%\subsubsection{Carves}
%Test

\monsterheader{assets/ext/icons/moofah}{Moofah}{\FiveStarOpen}{No Stars}
\textit{Small herbivores that can be found on the Deserted Island. Their soft, high-quality fur has been used since ancient times in clothing and ceremonial tools, and can be shaved off of them with cutting weapons. They are docile and raised as livestock in many regions.}

\monsterheader{assets/ext/icons/rajang}{Rajang}{\FiveStar\FiveStar\FiveStar\FiveStar\FiveStar\FiveStar\FiveStar\FiveStar}{8 Stars}
\textit{An ultra-aggressive creature that is rarely sighted and seldom survived. Survivors report it exhibits a strange attack. The Rajang is said to be a loner, and this isolated life has made it difficult to pin down its territorial leanings.}

\section{Brute Wyverns}
In addition to regular brute wyverns, there are also pseudo-wyverns in this category\hbNone whereas true brute wyverns walk only on their hind legs, pseudo-wyverns in this category walk on four legs but also don't have wings.

\monsterheader{assets/ext/icons/aptonoth}{Aptonoth}{\FiveStarOpen}{No Stars}
\textit{Relatively docile herbivores with characteristic crest plates. They form herds and raise young communally, and have been used as pack animals for generations. Their meat is tasty and nutritious, and they're very cautious around large monsters.}

\monsterheader{assets/ext/iconsL/deviljho}{Deviljho}{\FiveStar\FiveStar\FiveStar\FiveStar\FiveStar\FiveStar\FiveStar}{7 Stars}
\textit{The dreaded, nomadic Deviljho have no specific territory of their own. Their muscles swell if provoked, revealing old wounds. They need to feed constantly due to high body heat and can hunt nearby animals to extinction.}\sizecrowns{1,842}{2,619}%{1,803}{4,097} wow

\subsubsection{Physiology \& Ecology}
Deviljho is among the largest monsters outside of Elder Dragons. It is a bipedal brute wyvern characterized by its very muscular upper body, spiked chin and tiny forelegs. In addition to the spines on its chins, there are several rows of similar spines running down the length of its body, all the way to the tail. Its mouth may have developed from something similar to an otter's beak, but the resemblance is only fleeting when compared to the horrible maw that it is today. It has multiple rows of teeth and extremely strong jaws, making its bite perhaps the strongest of all wyverns. In addition, its lower jaw is covered in sharp spikes, allowing it to use its head like a club, which helps to make up for its practically useless front legs.

And yet, neither the teeth nor the spikes may be the most dangerous thing coming from Deviljho's maw: It's saliva is so acidic that it has been known to dissolve even hunter armor. Even if a hunter survives a bite, they may find their equipment damaged beyond repair. Just standing under its head carries a serious danger of having your helmet dissolved.

Deviljho will nearly always assume the role of apex predator in any region that it enters. However, this position is always contested and Deviljho carries the scars of dozens of battles for dominion over some other wyvern's territory. That it persists despite the many challengers is a testament to its savagery. The scars are especially well visible when it pumps itself up to fight at full power, something it will only do when it feels seriously threatened. When it does so, its shoulder muscles swell considerably and take on a bright red colour, contrasting the pink scars.

As if it didn't have enough ways of killing, it is capable of producing a mysterious flame-like emission which it shoots out of its mouth. This strange substance contains the Dragon Element and will decimate anything it comes in contact with. Some say that this ability betrays a Dragon heritage and proves that Deviljho is at least part Dragon.

\subsubsection{Behaviour}
Deviljho is a walking legend. Nearly everyone knows stories about the monster that even makes short work of the wyvern royalty: Rathian and Rathalos and most stories don't even have to exaggerate. Due to its nomadic nature, it could strike anywhere, at any time, adding to the terror it commands. Due to the many stories, it is difficult to seperate fact from fiction. And while the Guild has undertaken much effort to ascertain some facts about it, other parts of Deviljho's life remain clouded in mystery, such as how they reproduce and whether there is a difference between male and female Deviljho. It will normally attack anything and everything, except for certain times, which it signals by secreting a special hormone. It is believed that that signals its readiness to mate and that the hormone tells other Deviljho which might be in the area that it will not attack. Thankfully, there are only very rarely two Deviljho in the same area.

Its massive size and violent temper causes it to consume huge amounts of energy. As a result, Deviljho is always in search of food. It has even been known to be cannibalistic, attacking, killing and eating other Deviljho or even eating its own tail if it has been cut off. The adage "Is Deviljho always hungry?" is a common folk saying, a rhetorical question in response to a question whose answer is an emphatic yes.

Deviljho has no fixed territory, it is completely nomadic and has no particular preference or limitation of ecosystem. It has been sighted in tundras and near volcanoes as well as in jungles and swamps. It seems that deserts are the only biome it will not visit. The arrival of a Deviljho invariably disrupts the ecosystem, as it begins hunting predators and prey alike. A skilled observer can tell the presence of a Deviljho without ever seeing it or finding tracks, simply by observing the behaviour of other large wyverns.

\begin{hbNarrowTable}{Body Carves}{RY}
\textbf{d100} & \textbf{Carve}\\
01-35 & \tableicon{scale-dark-green} Deviljho Scale\\
36-63 & \tableicon{hide-dark-green} Deviljho Blackpiel\\
64-83 & \tableicon{claw-dark-green} Deviljho Ripper\\
84-94 & \tableicon{claw-dark-green} Deviljho Hardfang\\
95-98 & \tableicon{monster-jewel-dark-green} Deviljho Gem\\
99-100 & \tableicon{mantle-green} Deviljho Crook\\
\end{hbNarrowTable}

\begin{hbNarrowTable}{Tail Carve}{RY}
\textbf{d100} & \textbf{Carve}\\
01-60 & \tableicon{carapace-dark-green} Deviljho Flail\\
61-92 & \tableicon{scale-dark-green} Deviljho Scale, \tableicon{carapace-dark-green} Deviljho Flail\\
93-97 & \tableicon{monster-jewel-dark-green} Deviljho Gem, \tableicon{carapace-dark-green} Deviljho Flail\\
98-100 & \tableicon{mantle-green} Deviljho Crook, \tableicon{carapace-dark-green} Deviljho Flail
\end{hbNarrowTable}

\begin{hbNarrowTable}{Wound Head}{RY}
\textbf{d100} & \textbf{Carve}\\
01-50 & \tableicon{claw-dark-green} Deviljho Hardfang\\
51-85 & \tableicon{carapace-dark-green} Deviljho Scalp\\
86-100 & \tableicon{claw-dark-green} Deviljho Hardfang$\times$2
\end{hbNarrowTable}

It has been well-documented that Deviljho never shies away from a fight and will actively attack even the top predators of a region, but even it overreaches itself from time to time. It has been reported that Deviljho has attacked Rajang and even Elder Dragons. While that would certainly be a worthwile spectacle, a hunter would be well-advised to stay away from this fight, as it is likely one that even Deviljho will lose.


\monsterheader{assets/ext/icons/larinoth}{Larinoth}{\FiveStarOpen}{No Stars}
\textit{These giant herbivores are peaceful towards hunters unless their young are threatened. Their long necks let them eat hard-to-reach leaves and nuts, which they may drop if attacked while feeding, and eat constantly to maintain their size. They also have a unique sound-producing organ.}

\monsterheader{assets/ext/icons/zinogre}{Zinogre}{\FiveStar\FiveStar\FiveStar\FiveStar\FiveStar}{5 Stars}
\textit{Fanged wyverns whose bodies are streaked with electricity. Sharp claws and strong limbs allow them to thrive in mountainous terrain. During hunts, they gather numerous Thunderbugs to boost their power and enter a supercharged state.}\sizecrowns{1,309}{1,790}

\subsubsection{Physiology \& Ecology}
Zinogre used to be much rarer than it is today, only appearing in a small area around the mountains known as the Sacred Pinnacle near Yukomo Village. In recent years, Zinogre has been spotted further and further away. The exact reason is unknown, but it is believed to have been driven from its original habitat by something.

It is a four-legged monster with some superficial features similar to a wolf. Due to its atypical four-legged stance, as well as having fur, it is often falsely believed to be a beast, when it is actually a wyvern. Like the flying wyverns, Zinogre's front legs are much stronger than the hind legs. Unlike flying wyverns, it has not developed from a winged species, which is why it is classified as a four-legged brute wyvern, rather than a bird wyvern.

An observer will first note Zinogre's imposing posture. When not stalking prey, it holds its long, wolf-like head up high, displaying its dual horns. A closer observation reveals that  in addition to the amber ridges and white hair, Zinogre does indeed have turquoise scales, further evidence of its wyvern heritage. Its tail accounts for about half of its body length and is unusually flat. It serves as an additional leg when jumping, enabling Zinogre to make extremely powerful leaps.

Zinogre's main physical attacks are its massive claws, slams with its broad chest or a full somersault into a devastating tail strike. However, all of these attacks are empowered by its ability to store the Thunder Element in its hairs. This charge accumulates naturally over time and Zinogre can expend it to add an electric discharge to its attacks. This is sufficient for its prey, which is usually defeated by a single massive blow from its arm, but not for an extended fight with hunters. During such a fight, Zinogre draws on the rare Fulgurbug, a species of insect with which it has entered into a symbiotic relationship. The bugs, which can usually only be found in the vicinity of a Zinogrem, gather on its back. They carry the Thunder Element, recharging Zinogre rapidly.

In fact, attracting enough Fulgurbugs allows Zinogre to enter a super-charged state. In this state, Zinogre has gathered more energy than it could normally hold and it can be seen glowing and arcing from its back. Every attack it makes is paired with a massive energy surge and is sure to make short work of anything unfortunate enough to stand in its path. This level of energy is difficult for Zinogre to control and it takes some concentration to hold it. Usually, a Zinogre has little cause to make use of this energy-intensive and potentially dangerous technique. However, with every time it does, its experience with it will increase, making Zinogres that have survived hunter encounters particularly dangerous due to their improved mastery of their supercharged state.

Zinogre prefers mountainous terrain similar to its ancestral homeland, but has been known to visit hot, humid areas such as the tropical rainforest. This is believed to benefit its gathering of electric energy.

\subsubsection{Behaviour}
Zinogre is a solitary hunter and only territorial while raising its young. Before reaching adulthood, a young Zinogre is vulnerable. For this reason, Zinogre eschews its usual roaming solitary behaviour and instead works together with others to protect their young until they are old enough. When the young are grown up, the gathering will go their seperate ways and often not meet again for many years.

When charging up, Zinogre draws on its special relationship with the Fulgurbug. The benefit the latter provides is repaid with protection from the Fulgurbug's main predator: The Gargwa, which happen to be Zinogre's favourite meal. When Zinogre is in danger, it will attract Fulgurbugs to its back by means unknown. Though untested, ways to remove the bugs from the fur on its back might be effective in halting its charge.

\begin{hbNarrowTable}{Body Carves}{RY}
\textbf{d100} & \textbf{Carve}\\
01-35 & \tableicon{shell-teal} Zinogre Cortex\\
36-55 & \tableicon{hide-white} Zinogre Electrofur\\
56-75 & \tableicon{carapace-teal} Zinogre Shocker\\
76-92 & \tableicon{claw-teal} Zinogre Claw\\
93-97 & \tableicon{scale-teal} Zinogre Plate\\
98-100 & \tableicon{monster-jewel-teal} Zinogre Jasper\\
\end{hbNarrowTable}

\begin{hbNarrowTable}{Tail Carves}{RY}
\textbf{d100} & \textbf{Carve}\\
01-70 & \tableicon{carapace-teal} Zinogre Tail\\
71-97 & \tableicon{carapace-teal} Zinogre Tail, \tableicon{shell-teal} Zinogre Cortex\\
98-100 & \tableicon{carapace-teal} Zinogre Tail, \tableicon{scale-teal} Zinogre Plate\\
\end{hbNarrowTable}

\begin{hbNarrowTable}{Wound Head}{RY}
\textbf{d100} & \textbf{Carve}\\
01-70 & \tableicon{claw-teal} Zinogre Horn\\
71-84 & \tableicon{claw-teal} Zinogre Horn$\times$2\\
85-97 & \tableicon{shell-teal} Zinogre Cortex\\
98-100 & \tableicon{monster-jewel-teal} Zinogre Jasper\\
\end{hbNarrowTable}

\begin{hbNarrowTable}{Wound Leg}{RY}
\textbf{d100} & \textbf{Carve}\\
01-70 & \tableicon{claw-teal} Zinogre Claw\\
71-90 & \tableicon{claw-teal} Zinogre Claw$\times$2\\
91-100 & \tableicon{carapace-teal} Zinogre Shocker\\
\end{hbNarrowTable}

\section{Bird Wyverns}
Bird wyverns are characterised by their bipedal stance and slender build. Many species of bird wyvern have developed wings and are capable of flight. Those that can fly are sometimes referred to as \emph{true bird wyverns} and those that cannot as \emph{pseudo bird wyverns}.

\monsterheader{assets/ext/icons/gargwa}{Gargwa}{\FiveStarOpen}{No Stars}
\textit{Fightless, plain-dwelling bird wyverns with vestigial wings. Quite timid, Gargwa have been known to lay eggs when other creatures surprise them from behind. They are raised as livestock in numerous villages.}

\monsterheader{assets/ext/icons/jaggi}{Jaggi}{\FiveStar}{One Star}
\textit{Highly social, carnivorous bird wyverns that live in large packs. Young Jaggi males hunt in groups when attacking larger animals and have been known to steal wyvern eggs. Research suggests they operate under orders from a single alpha male.}

\monsterheader{assets/ext/iconsL/jaggia}{Jaggia}{\FiveStar}{One Star}
\textit{Female Jaggi that stay in packs, Jaggia generally live near the nest and are responsible for defending it and raising young. Smaller than mature males, but larger and tougher than the pack's countless young males. They also operate under orders from the alpha.}

\monsterheader{assets/ext/iconsL/great-jaggi}{Great Jaggi}{\FiveStar\FiveStar}{2 Stars}
\textit{The alpha male that leads Jaggi packs. Most males leave the group upon reaching maturity, returning late to compete with others. The dominant male then becomes a Great Jaggi. Apparently, they can issue fairly complex orders via howling.}\sizecrowns{840}{1,149}

\subsubsection{Physiology \& Ecology}
Jaggi are among the most common predators and occur in many different ecosystems. The two-legged bird wyverns are notable for their orange and lavender coloration, as well as the promintent frills that the males display. Their main weapons are their canine teeth and barbed tails. However, their most important strength is their pack, giving them a numbers advantage over prey they would otherwise have no chance of hunting.

The distinctive ear frills, which are prized trophies, are used for intimidation behaviour both inside the pack and towards prey. This is indicative of their highly developed social nature. A typical Jaggi pack will contain only a single fully grown male, all other males are adolescent and eager to prove themselves. It is often thought that Jaggia (the females) are larger than the males. And for the adolescent males in the pack, this is true. But this is merely an early stage in their lifecycle. The Jaggi which survive to adulthood grow up into the significantly larger Great Jaggi.

Besides being much larger, the Great Jaggi's flanks are fully lavender colored, only a stripe of the old orange coloration remains. In addition, the middle claw on each foreleg has developed into a fierce hook, useful for dismembering larger prey and ensuring the Great Jaggi always gets the best parts of slain prey.

The female Jaggia are larger and more massive than the adolescent males, though they are still dwarfed by the alpha male. They do not use their frills for threatening and they are less developed than in the males, hanging down limply on the sides of their heads, similar to dog ears. They also lack the tail barbs. If threatened, they can still be dangerous, because they have learned to use their higher body mass to their advantage, utilising poweful slams.

When alone, a Jaggi will largely eat fish and other small prey. When hunting as a pack, their favourite prey are Aptonoth. When hunting this large prey, they will attempt to confuse it with their numbers and fast movement, as well as sorrounding it, attacking from blind spots and causing bleeding wounds.

\subsubsection{Behaviour}
The rule of the Great Jaggi is absolute. No other male from the pack can even hope to challenge him, because he is the only fully grown male. When a Jaggi reaches adulthood, it is driven from the pack by the Great Jaggi and ventures into the wild, competing with the leaders of other packs for a position as Great Jaggi. Many Jaggi never manage to win a challenge for their own pack and sometimes a challenger is left mortally wounded.

The Jaggia are much more docile and will usually stay within their own pack. When encountered in the wild, they are often seen lazily sunbathing and can sometimes even be approached if one is brave. That said, during the time of the year when the Jaggia lay their eggs and the younglings hatch, the Jaggia will defend the nest and will be much less approachable.

The young Jaggi are often inexperienced or overeager to prove themselves. When facing an opponent, they are just as likely to posture and threaten than they are to actually attack, and will be easily demoralised by any real resistance. This changes when they are backed up (or driven by) the presence of the Great Jaggi. He will coordinate and strengthen his pack in a fight. The Great Jaggi is able to give surprisingly precise and complex orders via howling and other noises.

Normally, it will be the job of the Jaggia to protect the nest, the Jaggi are looking for prey and alerting the pack when a target has been found and the Great Jaggi will protect the pack's territory against other large predators. Unlike many other small wyverns, Jaggi have been observed attacking monsters such as the Yian Kut-Ku and Rathian, in order to drive them away from their territory.

Like many other small bird wyverns, Jaggi are afraid of fire.

\begin{hbNarrowTable}{Jaggi/Jaggia Carves}{RY}
\textbf{d100} & \textbf{Carve}\\
01-35 & \tableicon{claw-purple} Jaggi Fang\\
36-70 & \tableicon{bone-blue} Avian Finebone\\
71-88 & \tableicon{scale-purple} Jaggi Scale\\
89-100 & \tableicon{hide-purple} Jaggi Hide\\
\end{hbNarrowTable}

\begin{hbNarrowTable}{Great Jaggi Carves}{RY}
\textbf{d100} & \textbf{Carve}\\
01-43 & \tableicon{hide-purple} Great Jaggi Hide\\
44-73 & \tableicon{claw-blue} Great Jaggi Claw\\
74-95 & \tableicon{sac-grey} Screamer Sac\\
96-100 & \tableicon{monster-jewel-blue} Bird Wyvern Gem\\
\end{hbNarrowTable}

\begin{hbNarrowTable}{Break Great Jaggi Head}{RY}
\textbf{d100} & \textbf{Carve}\\
01-70 & \tableicon{webbing-purple} King's Frill\\
71-82 & \tableicon{carapace-purple} Great Jaggi Head\\
83-97 & \tableicon{sac-grey} Screamer Sac\\
98-100 & \tableicon{monster-jewel-blue} Bird Wyvern Gem\\
\end{hbNarrowTable}


\monsterheader{assets/ext/icons/velociprey}{Velociprey}{\FiveStar}{One Star}
\textit{Aggressive, carnivorous monsters that often travel and hunt in packs\hbNone even master hunters are careful to not let themselves get surrounded by these predators.}

\monsterheader{assets/ext/icons/genprey}{Genprey}{\FiveStar}{One Star}
\textit{Bird wyverns with a distinctive green-and-orange striped hide, Genprey live in packs in the Dunes and Primal Forest. Their claws and large fangs contain a paralyzing neurotoxin that they use to stun prey.}

\monsterheader{assets/ext/icons/ioprey}{Ioprey}{\FiveStar}{One Star}
\textit{A vivid red species of small carnivores often found in subtropical zones. Sacs in their throats contain a powerful poison that slowly drains the Health of their prey.}

\monsterheader{assets/ext/icons/giaprey}{Giaprey}{\FiveStar}{One Star}
\textit{A species of small bird-like carnivores known to inhabit the snowy mountains. Their white skin is beautiful, but their temperament is not. They are known to spit ice at hunters, and attack in deadly packs.}

\monsterheader{assets/ext/icons/velocidrome}{Velocidrome}{\FiveStar\FiveStar}{2 Stars}
\textit{Alpha monsters that lead Velociprey packs. Larger than their brothers and with a more prominent crest, Velocidrome use their strong hind legs to leap at prey, pinning them with sharp claws before calling for others.}\sizecrowns{781}{1,067}%{590}{1,736}

\monsterheader{assets/ext/icons/gendrome}{Gendrome}{\FiveStar\FiveStar}{2 Stars}
\textit{Alpha monsters that lead Genprey packs. Larger than their brothers and with a more prominent crest, Gendrome use the venom in their highly evolved fangs and claws to paralyze their prey.}\sizecrowns{778}{1,063}%{615}{1,037}

\monsterheader{assets/ext/icons/iodrome}{Iodrome}{\FiveStar\FiveStar}{2 Stars}
\textit{Alpha monsters that lead Ioprey packs. Larger than their brothers and with a more prominent crest, Iodrome spit a poison that slowly saps the life from their prey. They are found primarily in subtropical zones.}\par\sizecrowns{820}{1,121}%{619}{1,641}

\monsterheader{assets/ext/iconsL/giadrome}{Giadrome}{\FiveStar\FiveStar}{2 Stars}
\textit{The alpha leader of a pack of Giaprey, the Giadrome sports a beautiful crest. Larger than a Giaprey, any hunter silly enough to encroach on its turf will be frozen in a hail of ice.}\sizecrowns{664}{908}%{649}{959}

\subsubsection{Physiology \& Ecology}
Velociprey are two-legged pack hunters who rely on their high speed and long claws, as well as strength in numbers to take down prey\hbNone usually young Aptonoth. Their slender build identifies them as a bird wyvern, even though their lack of wings places them in the category of pseudo bird wyverns. Notable across all species of Velociprey are the prominent crests, which are present in both sexes and play an important role in determining the pack order.

For attack, Velociprey rely on their unusual, seven-clawed hands as well as their razor-sharp teeth. The teeth in particular are fused with the beak and are extremely hard, withstanding even gunpowder blasts. They have great mobility, owing to their strong hind legs, which also allow them to make leaping attacks and retreats.

The leader of the pack, the Velocidrome is the individual with the largest crest. On Velocidrome, the midle hand claw as well as the foot claws are particularly developed. As a result, it uses leaping attacks far more aggressively than its subordinate brethren.

Many species of Velociprey have developed in different biomes. Some have developed venom, which is either spat at prey or delivered via the claws. Between them, the many related species of Velociprey inhabit practically every biome on the planet and are an ever-present danger when venturing out into the wilderness. In many areas, a Velociprey attack is a far more realistic risk than an attack by a flying wyvern.

While they compete for prey with Jaggi, they have found different solutions to the same evolutionary problem: Velociprey are about speed, whereas Jaggi develop for raw strength. In addition, Velociprey are more intelligent than Jaggi, whereas Jaggi are more social.

\subsubsection{Behaviour}
Unless when intruding on their nest, Velociprey will usually not attack humanoids unless they are desperate. The issue of Velociprey perfectly illustrate why the Guild must exist: Whenever Velociprey attack a village, a disturbance in the ecosystem can be made directly responsible, as that is not their normal behaviour. If hunting were not the carefully monitored activity that it is, such attacks would be far more common.

When cornering prey, Velociprey coordinate effectively, communicating with surprising efficacy through calls. Whenever possible, they will keep their distance and tire out their prey before going in for the kill, though they are more than well-equipped for an encounter in close quarters. It is likely that one encounters the Velocidrome first, as it constantly patrols the area sorrounding the pack's nest, looking for prey and intruders. If it is not already escorted, it will quickly call for reinforcements before taking on the would-be nest robber. The crest is the Velocidrome's symbol of power, if it is broken, it will lose its authority in the pack, possibly even leading to infighting to determine the new pack alpha.

Like many small predators, Velociprey are afraid of fire.

\begin{hbNarrowTable}{Velociprey Body Carves}{RY}
\textbf{d100} & \textbf{Carve}\\
01-42 & \tableicon{claw-blue} Velociprey Fang$\times$1d4\\
43-75 & \tableicon{scale-blue} Velociprey Scale$\times$1d4\\
76-100 & \tableicon{hide-blue} Velociprey Hide\\
\end{hbNarrowTable}

\begin{hbNarrowTable}{Velocidrome Body Carves}{RY}
\textbf{d100} & \textbf{Carve}\\
01-53 & \tableicon{claw-blue} Velocidrome Claw\\
54-84 & \tableicon{hide-blue} Velocidrome Hide\\
85-96 & \tableicon{scale-blue} Velocidrome Scale\\
97-100 & \tableicon{hide-blue} Bird Wyvern Gem\\
\end{hbNarrowTable}

\begin{hbNarrowTable}{Wound Velocidrome Head}{RY}
\textbf{d100} & \textbf{Carve}\\
01-40 & \tableicon{claw-blue} Velocidrome Fang$\times$1d4\\
41-90 & \tableicon{sac-grey} Screamer Sac, \tableicon{claw-blue} Velocidrome Fang$\times$1d4\\
91-100 & \tableicon{carapace-blue} Velocidrome Head\\
\end{hbNarrowTable}
Other species use the same distribution for their respective claws, hides, fangs, scales and hides. Swap \tableicon{sac-grey}~Screamer Sac with \tableicon{sac-yellow}~Paralysis Sac for Gendrome, \tableicon{sac-purple}~Poison Sac for Iodrome and \tableicon{sac-cyan}~Frost Sac for Giadrome.


\section{Flying Wyverns}
True wyverns have a heavier build than bird wyverns and all species have at some point developed wings. Contrary to what the name would have you believe, not all flying wyverns can fly. In some species, the wingarms have regressed to being forearms again.

\monsterheader{assets/ext/icons/khezu}{Khezu}{\FiveStar\FiveStar\FiveStar\FiveStar}{4 Stars}
\textit{Loathsome wyverns that dwell in caves and other dark places. All but blind, Khezu hunt by smell, using electric shocks to paralyze their prey before pouncing on them from the walls or ceiling.}\sizecrowns{784}{993}%{436}{1,225}
\subsubsection{Physiology \& Ecology}
For many years, Khezu was considered a myth, a mysterious creature supposedly living underground, its body one long series of disgusting anomalies and a silent hunter to be afraid of. It was not until some time after the Guild was founded and information about monsters was shared in a more organised fashion that the facts about Khezu were seperated from the fantasies. It is true that it is an elusive wyvern, living in cold areas, nearly entirely underground and in complete darkness.

Its body has adapted in various ways to this environment: It's scaleless, rubbery skin is nearly translucent, appearing grey with visible blood vessels and is covered by a slimy mucus which, together with a layer of fat, helps to keep Khezu warm even when it is not moving. Its eyes have almost completely regressed, making it practically blind. Instead, it relies mainly on smell and hearing to locate predators and prey, which it does with great efficacy. It is suspected that Khezu also possesses an additional sense, owing to its electric organ, which allows it to sense nearby electrical disturbances.

Its scaleless skin and electricity producing organ are by far not the only strange features that set Khezu apart from other wyverns. It has a very flexible extending neck, which seemingly ends directly with its mouth with no discernible head. It often uses this neck to surprise prey in caves while it hangs from the ceiling using its suction tail, as well as holding on with four feet\hbNone the wingarms have hands with a little flexibility.

Becides its mouth full of crushing fangs, hunters should watch out for its saliva, which is highly acidic. It's not as strong as that of Deviljho, which has been known to eat through armor, but it is a threat nevertheless. It will also make frequent use of its ability to project electric shocks around it, while it itself is fully shielded from such effects thanks to its thick rubber skin.

Khezu is a hemaphrodite, meaning the same individual is both male and female. When reproducing, Khezu will paralyze a medium-sized monster, such as a Popo and lay its eggs inside the monster while it is still alive. The eggs are partially incubated before Khezu deposits them, so they will hatch quickly and the young, called whelps will have their first meal immediately. A whelp's mouth is actually underdeveloped at birth, making it unable to feed normally. It actually uses its tail to drain blood from prey until its mouth has fully developed.

Khezu is a migratory wyvern. For most of the year, it lives in cold and snowy areas, but during the winter it will move to somewhat warmer latitutes. It will always prefer to be underground and near running water. Water is believed to be necessary to keep its skin damp, similar to an amphibian. Staying underground has several advantages for Khezu: Outside, it is vulnerable to larger, more dangerous wyverns as well as handicapped by its lack of sight. Underground, it has the advantage due its superiour other senses and it can use the ceiling to attack from above. It is unclear whether Khezu can fly. The overall body structure is more similar to that of Tigrex than to Rathian, giving credibility to the theory that they can't. They certainly have very little cause to do so underground.

\subsubsection{Behaviour}
Khezu has had only limited contact with humanoids, as it lives far away from places where humans or wyverians build their villages. Lynians have actually had a great deal more contact with this wyvern than the other intelligent races. Because its biology was such a mystery, for a long time hunters would actually seek out the illustrious Khezu, considering it something of a cryptozoid.

The only occasions where one might see a Khezu above ground are during its migration or when the food in its habitat is seriously scarce. For the reasons above, Khezu itself prefers not to go outside. Even though that limits the threat it poses to society, it has been hunted semi-frequently (legally and illegally) since the confirmation of its existence. Khezu whelps are considered a delicacy, even though the monsters were almost certainly hunted without Guild permission, as Guild only very rarely posts quests for non-adult monsters. There is also a fluid harvested from a Khezu's body known as Pale Extract, which is used as part of many medicines, adding to the demand for Khezu parts.

\begin{hbNarrowTable}[t]{Body Carves}{RY}
\textbf{d100} & \textbf{Carve}\\
01-30 & \tableicon{carapace-white} Pale Steak, \tableicon{hide-white} Flabby Hide\\
31-50 & \tableicon{bone-white} Pale Bone, \tableicon{hide-white} Flabby Hide\\
51-75 & \tableicon{potion-white} Pale Extract$\times$2\\
76-90 & \tableicon{sac-yellow} Electro Sac\\
91-96 & \tableicon{carapace-white} Khezu Special Cut\\
97-100 & \tableicon{monster-jewel-blue} Large Wyvern Gem
\end{hbNarrowTable}

\begin{hbNarrowTable}[t]{Wound Head}{RY}
\textbf{d100} & \textbf{Carve}\\
01-54 & \tableicon{hide-white} Flabby Hide\\
55-79 & \tableicon{potion-white} Pale Extract\\
80-94 & \tableicon{carapace-white} Pale Steak\\
95-100 & \tableicon{carapace-white} Khezu Special Cut
\end{hbNarrowTable}

Of course, we cannot condone the unsanctioned hunting of monsters and would-be poachers should certainly not make a mistake and consider Khezu an easy target. In its native habitat, Khezu can move in complete silence and will likely know of the hunters long before they know of it. It will then attempt to attack from a surprise position, hanging from the ceiling or hiding in a crack. Instead of committing to an open fight it will more likely retreat and attempt to find another advantageous location.

\monsterheader{assets/ext/icons/rathian}{Rathian}{\FiveStar\FiveStar\FiveStar\FiveStar}{4 Stars}
\textit{Fire-breathing female wyverns, also known as the "Queens of the Land". With powerful legs and poison-secreting tails, they hunt mainly on the ground. Sometimes seen preying as a couple, Rathians and Rathalos cooperate well.}\sizecrowns{1,488}{2,034}%{1,151}{2,303}

\monsterheader{assets/ext/icons/rathalos}{Rathalos}{\FiveStar\FiveStar\FiveStar\FiveStar}{4 Stars}
\textit{Terrible wyverns called "Kings of the Skies". Along with Rathians, they stake wide territories centered around their nests. Rathalos descend on invaders from the sky, attacking with poison claws and breath of fire.}\par\sizecrowns{1,540}{2,104}%{1,140}{2,248}
\subsubsection{Physiology \& Ecology}
The Rath species is a well-known species of true flying wyvern inhabiting a wide range of environemnts. In a way, Rathian and Rathalos are the standard to which other flying wyverns are held. Body structure as well as strength of flying wyverns are always compared to the Raths. So, the physiology of Rathian and Rathalos warrants a more detailed description.

They share the overall body structure: The hind legs have four claws, one of which is an opposable thumb. The male's (Rathalos') claws are venomous, which it uses for swooping attacks from the air, similar to a hawk. Rathian on the other hand does not have venom in the claws, but rather in spikes on her tail, which suits her ground-based hunting style. The front legs serve purely as wings and have no hands or paws, only a one large finger and two thin fingers which are connected by webbing, as well as a pseudo-thumb which cannot move and simply serves as another spike in addition to the other ones facing the front on the wingarm. The wings have intricate, unique patterns which are used for attracting mates.

The tail accounts for about half of their body length and is equipped with a thick spiked club at the end. Their backs are covered in large, hardened scales, called shells, which go all the way to the tip of the tail. Along the length of the tail, they also protrude at the sides like saw-teeth. Both sexes possess a flame organ, the flame sac, though it is more developed in Rathian. Individuals vary greatly in how adept they are in its use\hbNone some Rath have developed a technique to incorporate the fire breath into a bite without hurting themselves. Some Rathian have been observed to have developed particularly explosive fireballs, a technique with which Rathalos evidently struggles.

\begin{hbNarrowTable}[t]{Body Carves}{RY}
\textbf{d100} & \textbf{Carve}\\
01-30 & \tableicon{scale-green} Rathian Scale/\tableicon{scale-red} Rathalos Scale\\
31-60 & \tableicon{shell-green} Rathian Shell/\tableicon{shell-red} Rathalos Shell\\
61-75 & \tableicon{sac-red} Flame Sac\\
76-88 & \tableicon{webbing-green} Rathian Webbing, \tableicon{webbing-red} Rathalos Webbing\\
89-97 & \tableicon{carapace-green} Rathian Spike, \tableicon{carapace-red} Rathalos Marrow\\
98-100 & \tableicon{mantle-green} Rathian Mantle, \tableicon{mantle-red} Rathalos Mantle
\end{hbNarrowTable}

\begin{hbNarrowTable}[t]{Tail Carves}{RY}
\textbf{d100} & \textbf{Carve}\\
01-60 & \tableicon{carapace-green} Rathian Tail/\tableicon{carapace-red} Rathalos Tail\\
61-97 & \tableicon{carapace-green} Rathian Tail, \tableicon{carapace-green} Rathian Spike/\tableicon{carapace-red} Rathalos Tail, \tableicon{carapace-red} Rathalos Marrow\\
98-100 & \tableicon{carapace-green} Rathian Tail, \tableicon{scale-green} Rathian Plate/\tableicon{carapace-red} Rathalos Tail, \tableicon{scale-red} Rathalos Plate
\end{hbNarrowTable}

Perhaps due to having to adapt to breathing fire, the head has lost all but a semblance of possessing a beak. The last remnand, a pointed upper jaw, is here to stay though, as it is how a young Rath escapes its egg. A Rath hatchling is unable to hunt for itself, though it is capable of digesting meat. Rathian will hunt for them and present the meat to the hatchlings impaled on the spike on her chin, a feature with Rathalos lacks.

Due to the widespread occurrence of the Rath species, there is some regional variety, from minor changes in appearance to the subspecies Pink Rathian and Azure Rathalos. Generally speaking, Rathian is more common and inhabits a wider range of environments, though they will seek out the somewhwat rarer Rathalos territories when it is time to mate.

\subsubsection{Behaviour}
In bringing up their young, Rathian and Rathalos have clear roles: Rathian takes care of and feeds the young, while Rathalos guards the territory and fights off threats to the hatchlings. During this time, Rathalos will be at his most aggressive. This is also the only time during which Rathian and Rathalos will live together. As soon as the hatchlings can fly and hunt for themselves, they will go seperate ways and Rathian will return to its territory, as it is usually her who has come to seek out the male, although it is Rathalos who has to impress the females with the patterns on its wings as well as displays of strength. Fighting is not common in these displays, as competing males live in seperate territories.

The Guild has a great deal of experience in regulating the population of Rathian and Rathalos and they are perhaps the most well-researched large monsters of all. That does not mean that they are easy marks for a hunt, however. When the Guild decides to post a quest for either one, there are always many hunters who are interested, but few who will be accepted to take the quest. Hunting your first Rathian or Rathalos is considered the trial by fire (no pun intended) among hunters, marking the transition between a hunter out of necessity, protecting one's home and the sorrounding area and a professional, full-time hunter who will travel the world, actively seeking out quests in order to take on the biggest and most dangerous monsters.

However, many have made the mistake of underestimating the wrath of the Raths, which can only lead to a failure in hunting them. It is not by accident that they are called "The Queen Of The Land" and "The King Of The Skies", respectively. In most areas, they are the top of the food chain. There are other, stronger wyverns, but they are much rarer or exist in different environments than the far more common Rathian, so the majority of villages come to understand and accept that they live under a queen. The Guild usually works to maintain this status quo, because it maintains a balance in the ecosystem which would be disturbed if Rathian were suddenly gone. This would ultimately be to the detriment of those who live there.

\begin{hbNarrowTable}[t]{Wound Wing}{RY}
\textbf{d100} & \textbf{Carve}\\
01-70 & \tableicon{claw-red} Rath Wingtalon\\
71-100 & \tableicon{webbing-green} Rathian Webbing/\tableicon{webbing-red} Rathalos Webbing
\end{hbNarrowTable}

\begin{hbNarrowTable}[t]{Wound Head}{RY}
\textbf{d100} & \textbf{Carve}\\
01-70 & \tableicon{shell-green} Rathian Shell/\tableicon{shell-red} Rathalos Shell\\
71-95 & \tableicon{sac-red} Flame Sac\\
96-100 & \tableicon{monster-jewel-green} Rathian Ruby/\tableicon{monster-jewel-red} Rathalos Ruby
\end{hbNarrowTable}

\monsterheader{assets/ext/icons/tigrex}{Tigrex}{\FiveStar\FiveStar\FiveStar\FiveStar\FiveStar\FiveStar}{6 Stars}
\textit{Flying wyverns whose primitive origins are obvious. Prone to violence, they display incredible ferocity with their claws, jaws, and developed limbs. They inhabit a wide area searching for prey, and have even been spotted in regions of harsh cold.}\sizecrowns{1,561}{2,134}%{1,388}{2,603}
\subsubsection{Physiology \& Ecology}
Tigrex is considered by many to be the original flightless flying wyvern and a prime example of how the wingarms of some species of wyverns have regressed to forearms, giving Tigrex very good maneuverability and speed on the ground. Another distinctive feature is its yellow and cyan striped coloring, which can light up red when Tigrex is fighting with all its strength. Its wingspan is not sufficient to allow Tigrex to fly, but the webbing, combined with its high jumping strength, allows it to travel relatively long distances by gliding. When charging, it can accelerate to speeds of up to 50\,km/h, leading into either a massive slam with its forearms or a bite from its fearsome maw. From a running start, Tigrex is capable of making astonishing leaps. Many a hunter has thought themselves in a safe position, only for Tigrex to demonstrate that they were very much not out of reach. Like other wyverns, Tigrex is capable of walking on only its hind legs and will assume this stance when not in combat.

Tigrex prefers areas with cliffs and crevices that it can use as ramps to swoop down on prey or corner its quarry. It is very adaptable to different climates and has been spotted in areas ranging from deserts to glaciers, though it prefers warmer climate, only seeking out cold areas to hunt its favourite food: Popo. Tigrex has impressive stamina. It is able to stalk prey for days while using very little energy. Among hunters, it is more well known for the all-out hyper-aggressive fighting style it will assume when in combat, which will leave it quickly exhausted and hungry.

Another weapon in Tigrex' arsenal is its unique roar. Rather than being a high-pitched warning sound like for other wyverns, it is a weapon in its own right. With its roar alone, Tigrex is able to shatter rocks and blow away nearby monsters and hunters. The roar is always preceeded by a huge inhale, giving a hunter just enough warning to drop to the ground and cover their ears, or suffer Tigrex' projected wrath.

\subsubsection{Behaviour}
Tigrex is not territorial. It wanders the land in search of prey and will knowingly enter another wyvern's turf. Most other wyverns will not challenge it while it is in their territory, because they (rightly) fear it. Until it feels genuinely threatened, Tigrex will attempt to use as little energy as possible, adopting a stalking hunting style. In this way, it chooses a target and then proceeds to pursue this target with a single-minded determination. When the prey has been cornered, it launches into its signature charge, instantly going from silent predator to maximum power. With Tigrex, there is usually no middle ground.

When stalking prey, Tigrex can sometimes be distracted, leaving it open for a suprise attack. When threatened, its first response is to roar. This alone is enough to cause most wyverns to flee. Before roaring, Tigrex will always assume a four-legged, stable stance, followed by the inhale. If it is unable to assume such a stance, because of restricted space or a restrained or injured forearm, it will not roar.

\begin{hbNarrowTable}{Body Carves}{RY}
\textbf{d100} & \textbf{Carve}\\
01-40 & \tableicon{scale-yellow} Tigrex Scale\\
41-77 & \tableicon{claw-yellow} Tigrex Claw\\
78-97 & \tableicon{shell-yellow} Tigrex Carapace\\
98-100 & \tableicon{mantle-yellow} Tigrex Mantle
\end{hbNarrowTable}
\begin{hbNarrowTable}{Tail Carves}{RY}
\textbf{d100} & \textbf{Carve}\\
01-97 & \tableicon{carapace-yellow} Tigrex Tail\\
98-100 & \tableicon{mantle-yellow} Tigrex Mantle, \tableicon{carapace-yellow} Tigrex Tail
\end{hbNarrowTable}
\begin{hbNarrowTable}{Wound Head}{RY}
\textbf{d100} & \textbf{Carve}\\
01-40 & \tableicon{shell-yellow} Tigrex Carapace\\
41-60 & \tableicon{shell-yellow} Tigrex Carapace, \tableicon{claw-yellow} Tigrex Fangs$\times$2d4 \\
61-92 & \tableicon{carapace-yellow} Tigrex Scalp, \tableicon{claw-yellow} Tigrex Fangs$\times$2d4\\
93-100 & \tableicon{shell-yellow} Tigrex Carapace, \tableicon{carapace-yellow} Tigrex Maw
\end{hbNarrowTable}
\begin{hbNarrowTable}{Wound Forearm}{RY}
\textbf{d100} & \textbf{Carve}\\
01-50 & \tableicon{claw-yellow} Tigrex Claw$\times$1d4\\
51-97 & \tableicon{shell-yellow} Tigrex Carapace, \tableicon{claw-yellow} Tigrex Claw$\times$1d4\\
98-100 & \tableicon{mantle-yellow} Tigrex Mantle, \tableicon{claw-yellow} Tigrex Claw$\times$1d4
\end{hbNarrowTable}

\section{Leviathans}
Leviathans are aquatic or at least partially aquatic predators. Like wyverns on land, they are typically the top of the food chain in the environment they are found in. Some Leviathans have developed the ability to swim in sand instead of water and are particularly adapted towards that environment, though they often have similar behaviour to their aquatic cousins.

\monsterheader{assets/ext/iconsL/delex}{Delex}{\FiveStar}{1 Star}
\textit{Carnivorous desert monsters that travel in schools of four or five. Delex often follow large predators in the hope of scavenging leftovers, and have even been known to hitch rides on hunting sandskiffs. They are extremely sensitive to sound.}

\monsterheader{assets/ext/iconsL/fish}{Fish}{\FiveStarOpen}{No Stars}
\textit{This category includes carnivorous fish (Sharq, Catfish), migratory fish (Tuna, Arowanas), and natatorial fish (Molids, Perciformes). Fish often flee when attacked, so hunters prefer to use harpoons to obtain materials from them.}

\monsterheader{assets/ext/icons/ludroth}{Ludroth}{\FiveStar}{1 Star}
\textit{Aquatic female monsters. Ludroth form "harems" around large males, gathering in territories designated as breeding grounds. They're known to be extremely aggressive towards outsiders, so caution is advised when treading in their territory.}

\monsterheader{assets/ext/icons/plesioth}{Plesioth}{\FiveStar\FiveStar\FiveStar}{3 Stars}
\textit{Giant piscine wyverns which can be spotted near bodies of water. Where wings would be found on other wyverns, it has developed fins specialized for swimming, and, as a result, cannot fly. Despite its fish-like appearance, it is just as comfortable on land.}

\section{Elder Dragons}
The greatest and most ancient force among the monsters are the Elder Dragons. A single one can pose a threat to even the larger cities of the world and Dundorma has actually been destroyed by one several times. Being the awesome forces that they are, it is no surprise that they are represented in many songs and myths.

Biologically speaking, not much is known about the Elder Dragons, as there is no known event of one being killed (and thus studied). In some cases (such as Kirin), the label "Elder Dragon" is simply attached to a monster that fits no other categorisation and is not considered definitive. It is not known whether the Elder Dragons each represent a whole species or are unique individuals.

The Guild sees all Elder Dragons as threats that must be eliminated, but there is evidence that in the past, Elder Dragons were revered as gods, and some such cults continue to exist today.

So little is known with certainty about the Elder Dragons that we are not even sure if the creatures listed here are individuals or whole species. A first step to finding this out would be successfully slaying one. As this has never been accomplished, we cannot be sure.

\monsterheader{assets/ext/iconsL/jhen-mohran}{Jhen Mohran}{\FiveStarOpen}{unrated}
\textit{Rare ore can be mined from this enormous dragons' back; thus it is considered a prosperity symbol. It swallows vast amounts of organic material and is always surrounded by scavenging Delex, which sailors use to locate it.}

\monsterheader{assets/ext/icons/kirin}{Kirin}{\FiveStarOpen}{unrated}
\textit{Elder dragon so rarely sighted that little is known of its ecology. It's been said it is one with lightning itself, and that its body becomes clothed in pure electricity when it is provoked.}

\monsterheader{assets/ext/icons/kushala-daora}{Kushala Daora}{\FiveStarOpen}{unrated}
\textit{A metal plated dragon known as the tempest of wind. Eyewitnesses report violent storms alongside the dragon, and its wide range means towns may be attacked.}

\monsterheader{assets/ext/iconsL/lao-shan-lung}{Lao-Shan Lung}{\FiveStarOpen}{unrated}
\textit{A giant dragon few have seen and lived to tell the tale. When on the rampage it wreaks havoc on all in its path.}

\monsterheader{assets/ext/icons/fatalis}{Fatalis}{\FiveStarOpen}{unrated}
\textit{Stories of this legendary dragon date back to antiquity. Many skilled hunters have sought to challenge it, but none have ever returned. A monster shrouded in mystery\ldots}

\vspace{8pt}{\centering
{\color{chapter}\LARGE\mreaves The Legend of the Black Dragon}\\[4pt]
\itshape
When the world is full of wyverns\\
The legend is revived\\
Meat is eaten, Bone is crunched.\\
And blood is sucked up dry\\[6pt]

He burns the earth\\
And melts through iron\\
He boils the rivers\\
And mows down trees\\
He awakens the winds\\
And lights an inferno\\[6pt]

He is called Fatalis\\
The wyvern of destiny\\
He is called Fatalis\\
The wyvern of destruction\\
Call for help\\
Run for your lives\\
And don't forget to\\
Pray to the skies\\[6pt]

He is called Fatalis\\
The wyvern of destiny\\
He is called Fatalis\\
The wyvern of destruction\\
Fatalis, Fatalis\\
Heaven and Earth are yours\\
Fatalis, Fatalis\\
Heaven and Earth are yours\\}
