\hbRootChapter{Introduction}

\hbBottomLeftArt{.751}{.95}{.85}{assets/ext/guildmarm-corner}

\hbLettrine{W}{elcome, Hunter,} to the world of Monster Hunter! It is a wide open, sparsely populated land full of lush landscapes teeming with wildlife, for better or worse. Large cities are rare, you will find the small village nestled in a protected niche to be far more common. The world is a wild and dangerous place, teeming with monsters and ancient secrets.

\begin{hbNote}[t]
\subsubsection*{What does he know?}
\noindent As with any supplement, feel free to ignore anything and everything written here and have your setting work how you want it to. Hell, I don't follow half the advice in this book.
\end{hbNote}

Hunters deal with the monster population, but there are many more adventures to be found out in the wilderness. The few large cities are true marvels of the world, protected by ingenious defensive structures such as the battlequarters of Dundorma. The resourcefulness of the cities' engineers allows these few bastions to stand against the wild monsters and the far more devious dragons.

While I have tried to stick with the \MH{} Canon as much as possible, there is not a great deal of detail we get from the games themselves. I have filled in gaps in many places and added my own details. From the games we get a much better impression of the \emph{feeling} of the world rather than the exact details of the setting.

The feeling is this: The world is vast and alive, vibrant and colorful. It is nature at its most savage, and yet its inhabitants live in the moment and enjoy it, rather than fearing for tomorrow. Don't worry about getting the location of Dundorma wrong in your game. It is far more important to convey the feeling of the lush forests and endless plains, of something always lurking behind the next bush, on the prowl. The world is dangerous. The ladder of monsters \emph{starts} with velociraptors and goes all the way up to world eating dragons of the ancient times.

\subsection{Adaptation Pains}
\MH{} wasn't made with \DND{} in mind, so it is not surprising that some assumptions of the two systems don't match up. Here are some aspects that need adaptation and my take on how to reconcile them:

\subsubsection{Magic}
In the \MH{} canon, there are no spellcasters. There are certainly no hunters who primarily use magic to fight. Taking magic out of \DND{} takes too much out of the experience in my opinion. If you and your group don't want to use magic, don't. Here are some thoughts on it regardless:
\begin{itemize}
\item Magic users are rare. There are no institutions to learn magic, such as a magical college. An attempt to establish such a thing could be a campaign in its own right.
\item A wizard character would have learned his art directly at the hand of a master and might be expected to take an apprentice when he is experienced enough.
\item Spells that raise the dead or travel between planes are not learned normally. They may exist, but special measures must be taken to learn them.
\item Some (but not all) magic relates to spirits (see above). Perhaps instead of summoning an earth elemental, consider flavouring it as an earth spirit. When creating a flaming sword, one creates the sword and prepares it in such a way as to be a vessel for a fire spirit to inhabit. This provides an interesting explanation for why your sword \textit{must} be so richly decorated: It must provide a pleasing vessel for the spirit to come willingly. A more evil mage might bind a spirit to the sword against its will.
\item In addition to spellcasting magic, there is the field of alchemy. This need not have anything to do with the occult, but simply be a craft that exploits the special properties of the herbs or monster parts to make potions of seemingly magical effect. Or it could be magic.
\end{itemize}

\subsubsection{Religion}

\hbWideBottomArtFirstPage{1.784}{.892}{assets/ext/sojourn}

Be it clerical magic or literal divine intervention, the gods play an important role in \DND. The people in the Monster Hunter games don't typically make references to a god or gods, except when referring to certain monsters (mostly Elder Dragons). There might be people who see the Elder Dragons as gods, but no sane person would worship them. It is conceivable, however, that in some areas, people make sacrifices to avert the threat of an Elder Dragon attack. One could see the dragons as living creatures of legend, but some imbue them with divinity, which makes them a natural force and an integral part of the universe. That raises some question about the consequences of hunting them\ldots

As far as mainstream religion is concerned, the games do make references to spirits at times. I would suggest using the rules for spiritualism from the \DMG. I won't reproduce the text here, but essentially it describes the belief that everything is inhabited by a spirit. One doesn't worship one particular spirit but instead reveres and respects them all. A clerical domain represents a spirit who is particularly close to you, whom you might even know personally and whose powers you are able to wield specifically. Otherwise, you channel the powers of the combined spirit world. That means you are free to chose whatever cleric domain you wish.

\subsubsection{Miscellaneous}
Some short notes that didn't warrant their own categories:

\paragraph{Falling Damage.} There is no falling damage in \MH{}. Neither for players nor for Monsters. Let's see you abuse that.
\paragraph{Sand.} Sand is basically water for monsters that can swim in it. Given the right items, the players might be able to swim in it as well, but otherwise, it's solid ground for them and water for the monsters.
\paragraph{Holding Your Breath.} In \MH{}, everyone is Guybrush Threepwood and can hold their breath for ten minutes. I would go so far as to say everyone can just breathe underwater and save myself the bookkeeping.
\paragraph{Video Game Flair.} \MH{} is pretty comical. In a lot of places, I kept using the video game mechanics pretty much unchanged, usually to comedic effect, such as the falling damage. One other example would be additional drops from getting breaks on monster parts. It's very videogamey (why do I get the King's Frills from Great Jaggi if I \emph{break} the frills?) but it fulfills the same purpose it does for the video game: It rewards going for those risky breaks and getting the rare drop is sweet.
\hbWideBottomArtFirstPageFix
